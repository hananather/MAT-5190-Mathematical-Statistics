\begin{abstract}
    These are course notes for MAT 5190.
\end{abstract}

\section{Probability Theory Brief Review}
\subsection{Set Theory}

\begin{definition}[Sample Space]
The set of all possible outcome of an experiment, $S$ is called the \vocab{sample space} for the experiment.
\end{definition}
Note how the sample space entirely depends on the experiment. Once the sample space has been defined, we can consider \textit{collections} of possible outcomes of the experiment.
\begin{definition}[Event]
An \vocab{event} is a subset of the sample space or equivalently
an event is a collection of outcomes of an experiment.
\end{definition}
We say event $A \subseteq S$ occurs if the outcome of the experiment is in the set $A$. 

\vocab{Union}
\vocab{Complement}
\vocab{intersection}

\begin{theorem}
For any three events, A, B, and C, defined on a sample space S,
    \begin{enumerate}
        \item A
    \end{enumerate}
\end{theorem}

\begin{definition}[Disjoint or mutually exclusive]

\end{definition}

\begin{definition}[Partition]

\end{definition}
\subsection{Basics of Probability Theory}
\begin{definition}[Sigma algebra]

\end{definition}

\begin{definition}[Probability Function]

\end{definition}

\begin{theorem}
If $\PP$ is a probability function and $A$ 

\end{theorem}

\subsection{Conditional Probability and Independence}

\subsection{Random Variables}

\subsection{Distribution Functions}

\subsection{Density and Mass Functions}

\begin{question}
hello
\end{question}