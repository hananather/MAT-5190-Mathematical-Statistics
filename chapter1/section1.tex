\begin{abstract}
    These are course notes for MAT 5190.
\end{abstract}

\section{September 8, 2021}
\subsection{Set Theory}

\begin{definition}[Sample Space]
The set of all possible outcome of an experiment, $S$ is called the \vocab{sample space} for the experiment.
\end{definition}
Note how the sample space entirely depends on the experiment. Once the sample space has been defined, we can consider \textit{collections} of possible outcomes of the experiment.
\begin{definition}[Event]
An \vocab{event} is a subset of the sample space or equivalently
an event is a collection of outcomes of an experiment.
\end{definition}
We say event $A \subseteq S$ occurs if the outcome of the experiment is in the set $A$.
\textbf{Union}: 
$A\cup B = 
\{
x: x \in A \text{ or } x \in B
\}
$
\textbf{Complement}:
$A^c = 
\{
x: x \notin A
\}$.
\textbf{Intersection}:
$
A \cap B: 
\{
x: x \in A \text{ and } x \in B
\}
$

\begin{definition}[Disjoint or mutually exclusive, pairwise disjoint]
Two events $A$ and $B$ are \vocab{disjoint (or mutually exclusive)}  if $A \cap B = \emptyset$. The events $A_1, A_2, ...$ are pairwise disjoint if $A_i \cap A_j = \emptyset$ for all $i \neq j$.
\end{definition}

\begin{example}
    Disjoint sets are sets with no points in common. Basic example is the collection
    $$
    A_i = [i, i+1), i = 0,1,2,..
    $$
    Note the $\cup_{i=1}^{\infty}A_i = [0, \infty)$
\end{example}

\begin{definition}[Partition]
If $A_1, A_2,...$ are pairwise disjoint and $ \cup_{i=1}^{\infty}A_i = S$, then the collection $A_1, A_2,...$ forms a \vocab{partition} of $S$.
\end{definition}
\subsection{Basics of Probability Theory}
\begin{definition}[Sigma algebra]

\end{definition}

\begin{definition}[Probability Function]

\end{definition}

\begin{theorem}
If $\PP$ is a probability function and $A$ 

\end{theorem}

\subsection{Conditional Probability and Independence}

\subsection{Random Variables}

\subsection{Distribution Functions}

\subsection{Density and Mass Functions}

\begin{question}
hello
\end{question}