\begin{abstract}
    These are course notes for MAT 5190.
\end{abstract}



\section{Basic Concepts}
\subsection{Fields and $\sigma$-Fields}
\begin{definition}
    A class (set) of subsets of $S$ is said to be a field, and is denoted by $\SF$, if 
    \begin{enumerate}[(i)]
        \item $\SF$ is a non-empty class.
        \item $A \in \SF$ implies that $A^c \in \SF$ (closed under complementations)
        \item $A_1, A_2, \in \SF$ implies that $A_1 \cup A_2 \in \SF$ (that is, $\SF$ us closed under pairwise unions).
    \end{enumerate}
\end{definition}
Note that there are two key consequences of the definition of a field:
\begin{enumerate}
    \item $S, \emptyset \in \SF$
    \item If $A_j \SF, j =1,2,..,n$, then, $\bigcup_{j=1}^{k}A_{j} \in \SF$, $\bigcap_{j=1}^{k}A_{j} \in \SF$ for any finite $n$.
\end{enumerate}


\begin{definition}[Sigma algebra]
    A collection of subsets of $S$ is called a \vocab{sigma algebra} ( or \vocab{Borel field)}, denoted by $\SB$, if it satisfies the following three properties:
    
    \begin{enumerate}[(a)]
        \item $\emptyset \in \SB$ (the empty set is an element).
        \item If $A \in \SB$, then $A^c \in \SB$ (closed under complement) 
        \item If $A_1, A_2, ... \in \SB$, then, $\cup_{i=1}^{\infty}A_i \in \SB$ (closed under countable unions).
    \end{enumerate}
\end{definition}
Many different sigma algebras can be associated with a sample space $S$, the collection $\{\emptyset, S\}$ is the trivial sigma algebra. The only sigma algebra that we will be concerned with is smallest one that contains all open sets in a given sample space.

\subsection{Basics of Probability Theory}
\begin{definition}[Probability Function]
    Given a sample space $S$ and a sigma algebra $\SB$, a \vocab{probability function} is a function $\PP$ with domain $\SB$ that satisfies
    \begin{enumerate}
        \item $\PP(A) \geq 0$ for all $A \in \SB$
        \item $\PP(S) = 1$
        \item If $A_1, A_2,... \in \SB$ are pairwise disjoint, then $\PP(\cup_{i=1}^{\infty}A_i) = \sum_{i=1}^{\infty}\PP(A_i)$
    \end{enumerate}
\end{definition}


\subsection{Conditional Probability and Independence}
\begin{definition}[Conditional Probability]
    If $A$ and $B$ are events in S, and $\PP(B) \geq 0$, then the \vocab{conditional} probability of $A$ given $B$ is
    $$
    \PP(A|B) = \frac{\PP(A \cap B)}{\PP(B)}
    $$
\end{definition}
Note that in the calculation of conditional probability the event $B$ becomes the sample space. The intuition is that our original sample space $S$ has been updated to $B$. All further occurrences are then calibrated with respect to their relation with $B$.

\begin{theorem}[Bayes' Rule]

$A_1, A_2,...$ be a partition of the sample space, and let $B$ be any set. The for each $i = 1,2,...$

$$
\PP(A_i|B) = \frac{\PP(B|A_i)\PP(A_i)}{\sum_{j=1}^{\infty}\PP(B|A_j)\PP(A_j)}
$$
\end{theorem}

\begin{definition}[Independent, Mutually independent]
    Two events $A$ and $B$ are \vocab{statistically independent} if
    $$
    \PP(A \cap B) = \PP(A)\PP(B)
    $$
    A collection of events $A_1,...,A_n$ are \vocab{mutually independent} if for any subcollection $A_{i1},...A_{ik}$, we have 
    $$
    \PP\left( \bigcup_{j=1}^{k}A_{ij} \right ) = \prod_{j=1}^{k} \PP(A_{ij})
    $$
\end{definition}
\subsection{More Basics}
\begin{definition}[Random variable]
    A \vocab{random variable} is a function from sample space $S$ into $\RR$.
\end{definition}
With every random variable $X$ we associate a function called the cumulative distribution function of $X$.
\begin{definition}[cdf, pmf, pdf]
    The \vocab{cumulative distribution function} (cdf) of a random variable $X$, denoted by $F_{X}(x)$, is defined by 
    $$
    F_X(x) = \PP_X(X\leq x), \text{ for all } x.
    $$
    The \vocab{probability mass function} (pmf) of a \textit{discrete} random variable $X$ is given by
    $$
    f_X(x) = \PP_X(X=x) \text{ for all } x.
    $$
    The \vocab{probability density function} (pdf), $f_X(x)$ of a \textit{continuous} random variable is $X$ is the function that satisfies
    $$
    F_X(x) = \int_{-\infty}^{x}f_X(t)dt \text{ for all } x
    $$
    
    
\end{definition}
