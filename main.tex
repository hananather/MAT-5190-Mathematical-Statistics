
\documentclass[8.5pt,twoside=semi,openright,numbers=noenddot]{scrartcl}

\usepackage{amsmath,amssymb,amsthm}
\usepackage[usenames,svgnames,dvipsnames]{xcolor}

\usepackage{amsmath,amssymb,amsthm}
\usepackage{mathrsfs}
\usepackage[usenames,svgnames,dvipsnames]{xcolor}
\usepackage{hyperref}
\usepackage[nameinlink]{cleveref}
\usepackage{textcomp}
\usepackage{enumerate}
\usepackage{mathtools}
\usepackage{microtype}
\usepackage{geometry}
\usepackage[normalem]{ulem}
\usepackage{stmaryrd}
\usepackage{wasysym}
\usepackage{multirow}
\usepackage{prerex}
\usepackage{marginnote}
\usepackage{sidenotes}

\usepackage[headsepline ]{scrlayer-scrpage}
\renewcommand{\headfont}{}
\addtolength{\textheight}{3.14cm}
\setlength{\footskip}{0.5in}
\setlength{\headsep}{10pt}


\automark[section]{section}


\rohead{\footnotesize\thepage}
\rehead{\footnotesize \textbf{\sffamily MAT 5190}, Hanan Ather}
% More script letters etc.
\newcommand{\SA}{\mathcal A}
\newcommand{\SB}{\mathcal B}
\newcommand{\SC}{\mathcal C}
\newcommand{\SD}{\mathcal D}
\newcommand{\SE}{\mathcal E}
\newcommand{\SF}{\mathcal F}
\newcommand{\SH}{\mathcal H}
\newcommand{\SX}{\mathcal X}
\newcommand{\SY}{\mathcal Y}
\newcommand{\iid}{\text{ i.i.d.}}


\hypersetup{
    colorlinks,
    linkcolor={red!50!black},
    citecolor={green!50!black},
    urlcolor={blue!80!black}
}

\renewcommand*{\marginfont}{\color{red}\sffamily}

\newcommand{\II}{\mathbb I}
\newcommand{\CC}{\mathbb C}
\newcommand{\EE}{\mathbb E}
\newcommand{\FF}{\mathbb F}
\newcommand{\NN}{\mathbb N}
\newcommand{\QQ}{\mathbb Q}
\newcommand{\RR}{\mathbb R}
\newcommand{\PP}{\mathbb P}
\newcommand{\VV}{\mathbb V}

\newcommand{\ZZ}{\mathbb Z}



\newcommand{\eps}{\varepsilon}
\newcommand{\abs}[1]{\left\lvert #1 \right\rvert}
\newcommand{\norm}[1]{\left\lVert #1 \right\rVert}
\newcommand{\dang}{\measuredangle} %% Directed angle
\newcommand{\ray}[1]{\overrightarrow{#1}} 
\newcommand{\seg}[1]{\overline{#1}}
\newcommand{\arc}[1]{\wideparen{#1}}

\newcommand{\vocab}[1]{\textbf{\color{blue} #1}}
\newcommand{\term}[1]{\textit{\color{red} #1}}

\usepackage{thmtools}
\usepackage[framemethod=TikZ]{mdframed}
\usepackage{hyperref}
\usepackage[nameinlink]{cleveref}


\usepackage[headsepline]{scrlayer-scrpage}
\definecolor{crimsonglory}{rgb}{0.75, 0.0, 0.2}
\definecolor{denim}{rgb}{0.08, 0.38, 0.74}
\definecolor{egyptianblue}{rgb}{0.06, 0.2, 0.65}
\definecolor{lava}{rgb}{0.81, 0.06, 0.13}

\renewcommand*{\sectionformat}{\color{lava}\S\thesection\autodot\enskip}
\renewcommand*{\subsectionformat}{\color{Blue} \S\thesubsection\autodot\enskip}


\addtokomafont{partprefix}{\rmfamily}



\theoremstyle{definition}
\mdfdefinestyle{mdbluebox}{%
	roundcorner = 10pt,
	linewidth=1pt,
	skipabove=12pt,
	innerbottommargin=9pt,
	skipbelow=2pt,
	nobreak=true,
	linecolor=blue,
	backgroundcolor=TealBlue!5,
}
\declaretheoremstyle[
	headfont=\sffamily\bfseries\color{MidnightBlue},
	mdframed={style=mdbluebox},
	headpunct={\\[3pt]},
	postheadspace={0pt}
]{thmbluebox}


\mdfdefinestyle{mdredbox}{%
	linewidth=0.5pt,
	skipabove=12pt,
	frametitleaboveskip=5pt,
	frametitlebelowskip=0pt,
	skipbelow=2pt,
	frametitlefont=\bfseries,
	innertopmargin=4pt,
	innerbottommargin=8pt,
	nobreak=true,
	linecolor=RawSienna,
	backgroundcolor=Salmon!5,
}
\declaretheoremstyle[
	headfont=\bfseries\color{RawSienna},
	mdframed={style=mdredbox},
	headpunct={\\[3pt]},
	postheadspace={0pt},
]{thmredbox}

\mdfdefinestyle{mdgreenbox}{%
	skipabove=8pt,
	linewidth=2pt,
	rightline=false,
	leftline=true,
	topline=false,
	bottomline=false,
	linecolor=ForestGreen,
	backgroundcolor=ForestGreen!5,
}
\declaretheoremstyle[
	headfont=\bfseries\sffamily\color{ForestGreen!70!black},
	bodyfont=\normalfont,
	spaceabove=2pt,
	spacebelow=1pt,
	mdframed={style=mdgreenbox},
	headpunct={ --- },
]{thmgreenbox}
\declaretheoremstyle[
	headfont=\bfseries\sffamily\color{ForestGreen!70!black},
	bodyfont=\normalfont,
	spaceabove=2pt,
	spacebelow=1pt,
	mdframed={style=mdgreenbox},
	headpunct={},
]{thmgreenbox*}

\mdfdefinestyle{mdblackbox}{%
	skipabove=8pt,
	linewidth=3pt,
	rightline=false,
	leftline=true,
	topline=false,
	bottomline=false,
	linecolor=black,
	backgroundcolor=RedViolet!5!gray!5,
}
\declaretheoremstyle[
	headfont=\bfseries,
	bodyfont=\normalfont,
	spaceabove=2pt,
	spacebelow=1pt,
	mdframed={style=mdblackbox},
	headpunct={ --- }
]{thmblackbox}

\mdfdefinestyle{mdbluebox*}{%
	skipabove=8pt,
	linewidth=1.5pt,
	rightline=false,
	leftline=true,
	topline=false,
	bottomline=false,
	linecolor=Blue!70!black,
	backgroundcolor=TealBlue!5,
}
\declaretheoremstyle[
	headfont=\bfseries\sffamily\color{Blue!70!black},
	bodyfont=\normalfont,
	spaceabove=2pt,
	spacebelow=1pt,
	mdframed={style=mdbluebox*},
	headpunct={},
]{thmbluebox*}

\newenvironment{note}{%
	\begin{mdframed}[linecolor=Blue,backgroundcolor=TealBlue!5, roundcorner = 10pt, nobreak=true]%
	\color{black}}%
	{\end{mdframed}}






\declaretheorem[%
style=thmgreenbox*,name=Theorem,numberwithin=section]{theorem}
\declaretheorem[style=thmblackbox,name=Lemma,sibling=theorem]{lemma}
\declaretheorem[style=thmbluebox,name=Corollary,sibling=theorem]{corollary}
\declaretheorem[style=thmblackbox,name=Example,sibling=theorem]{example}
\declaretheorem[style=thmbluebox*,name=Definition,sibling=theorem]{definition}
\declaretheorem[%
style=thmbluebox*,name=Fact,numberwithin=section]{proposition}

\newenvironment{moral}{%
	\begin{mdframed}[linecolor=green!70!black]%
	\bfseries\color{green!50!black}}%
	{\end{mdframed}}

\newenvironment{fact}{%
	\begin{mdframed}[linecolor=black]}%
	{\end{mdframed}}

\newenvironment{question}{%
	\begin{mdframed}[linecolor=red!70!black]%
	\bfseries\color{RawSienna}%
	\color{black} \textbf{Question:}} 
	{\end{mdframed}}


\declaretheoremstyle[
	headfont=\bfseries\color{RawSienna},
	mdframed={style=mdredbox},
	headpunct={\\[3pt]},
	postheadspace={0pt},
]{thmredbox}

\title{Mathematical Statistics I}
\subtitle{MAT 5190}

\author{Hanan Ather}
\date{Fall 2021}





\begin{document}
\maketitle
\tableofcontents
\newpage

\begin{abstract}
    These are course notes for MAT 5190.
\end{abstract}




\section{September 8, 2021}
\begin{definition}
    A \vocab{probability function} denoted by \PP is a (set) function which assigns each event $A$ a number denoted by $\PP(A)$, called the probability of A, and satisfies the following requirements:
    
\end{definition}
\subsection{Bayes' Rule}
\subsection{Random Variables}
\subsection{Transformation of expectations}
\section{September 13, 2021}
\subsection{Discrete Distributions}


\section{September 15, 2021}
\subsection{Continuous Distributions}
\section{September 20, 2021}
\subsection{Exponential Family}
A family of pdf's (or pmf's) is called \textit{exponential family} if it can be written in the form
\begin{align}\label{exp_family}
    f(x|\theta) = h(x)c(\theta)\text{exp} \left \{ \sum_{i=1}^kt_i(x)w_i(\theta)\right \}
\end{align}
where $h(x) \geq 0, c(\theta) \geq 0, w_i(\theta) \in \RR, t_i(x) \in \RR$. Here the parameter $\theta = {\theta_1,...,\theta_d}$ is vector-valued. If $d=k$, the family is called a \textit{full exponential family}. If $d < k$, the family is called a \textit{curved exponential family}. 

\begin{example}[Binomial family with $n$ known]
    $$f(x|p) = {n \choose x} p^{x}(1-p)^{n} \left ( \frac{p}{1-p}\right )^x = 
    {n \choose x} (1-p)^{n}\text{exp} \left \{ xlog \left( \frac{p}{1-p}\right) \right \}$$
    $$
    =h(x)c(\theta)\text{exp} \left \{ \sum_{i=1}^kt_i(x)w_i(\theta)\right \}
    $$
\end{example}
\begin{example}[Poisson Family]
    $$
    f(x|\lambda) = \frac{e^{-\lambda}\lambda^{x}}{x!}
    =\frac{1}{x!}e^{\lambda}\text{exp}\{xlog\lambda\} =
    h(x)c(\theta)\text{exp} \left \{ \sum_{i=1}^kt_i(x)w_i(\theta)\right \}
    $$
\end{example}

\begin{theorem}
If $X$ is a random variable whose probability density function is given by \ref{exp_family} then
$$
\EE \left (
\sum_{i=1}^{k} = \frac{\partial w_i(\theta)}{\partial\theta_j}t_i(X)\right)
= -\frac{\partial}{\partial_j} \text{log} c(\theta)
$$

$$
\VV ar \left (
\sum_{i=1}^{k}  \frac{\partial w_i(\theta)}{\partial\theta_j}t_i(X)\right)
= -\frac{\partial^2}{\partial \theta_j} \text{log} c(\theta) 
- \EE \left (
\sum_{i=1}^{k} = \frac{\partial w_i(\theta)}{\partial^2 \theta_j}t_i(X)\right)
= -\frac{\partial^2}{\partial \theta_j^2} \text{log} c(\theta)$$
\end{theorem}
\subsection{Natural Parameterization}
If we use $\eta_i = w_i(\theta)$ in formula (\ref{exp_family}) and $\eta =(\eta_1,...,\eta_k)$, we obtain the \textit{natural parametrization}:

\begin{align}\label{natural}
    f(x|\eta) = h(x)c^{*}(\eta) \text{exp} \left \{ 
    \sum_{i =1}^{k}t_i(x) \eta_i
    \right \}
\end{align}
where $h(x)$ and $t_i(x)$ are the same as in formula (\ref{natural}). The natural space is $\SH = \{\eta: \int_{-\infty}^{\infty}\text{exp}\{ \sum_{i =1}^{k}t_i(x) \eta_i\}$. We have
$$
c^*(\eta) = \frac{1}{\int_{-\infty}^{\infty}\text{exp}\{ \sum_{i =1}^{k}t_i(x) \eta_i\}}, \quad \eta \in \SH
$$
\vocab{Normal Family.} The natural parametrization if $\eta_1 = 1/\sigma^2$, $\eta_2 = \mu/\sigma^2$ with natural parameter space $\SH = \{(\eta_1,\eta_2): 0 < \eta_1 < \infty, -\infty < \eta_2 < \infty \}$. We have $\mu = \eta_2/ \eta_1$ and $\sigma^2 = 1/\eta_1$ and hence the normal pdf 
$$
f(x|\mu, \sigma) = \frac{1}{\sqrt{2\pi}\sigma}\text{exp}\left \{ 
-\frac{\mu^2}{2\sigma^2}
\right \}
\text{exp}\left \{ 
-\frac{x^2}{2\sigma^2} + \frac{x \mu}{ \sigma^2}
\right \}
$$
can be written as,
$$
f(x|\eta_1, \eta_2) = \frac{\sqrt{\eta_1}}{\sqrt{2\pi}}
\text{exp}\left \{ 
-\frac{\eta_2^2}{2\eta_1}
\right \}
\text{exp}\left \{ 
-\frac{\eta_1x^2}{2} + \eta_2x
\right \}
$$
$$
= f(x|\eta) = h(x)c^{*}(\eta) \text{exp} \left \{ 
    \sum_{i =1}^{k}t_i(x) \eta_i
    \right \}
$$

\section{September 22, 2021}
\subsection{Location and Scale Families}
In this lecture we discussed three techniques for constructing families of distributions. Each of these technique relies on first specifying a single pdf,
$f(z)$, called the \textit{standard pdf} for the family. The other pdf's in the family are generated by applying a certain transformation to the standard pdf. 

\begin{theorem}
    Let $f(z)$ be any odf and $\mu$ and $\sigma >0$ be arbitrary constants. Then the following function is also a pdf:
    $$
    g(x|\mu, \sigma) = \frac{1}{\sigma}f \left ( 
    \frac{x-\mu}{\sigma}
    \right )
    $$
\end{theorem}
The family $\{ f(x-\mu): - \infty < \mu < \infty \}$ is called a \term{location family} with a standard pdf $f(z)$. The location parameter $\mu$ shifts the graph $f(z)$ with $\mu$, with out changing its shape.\\
\\
\vocab{Representation:} Let $Z$ be a random variable with pdf $f(z)$ and $X = \mu + Z$. Then the pdf of $X$ is $g(x|\mu)$. If $F_z(z)$ and $F_x(x)$ are cdf's of $Z$, respectively $X$, then
$$
F_X(x) = F_z(x-\mu).
$$
\vocab{Uniform location family.} By taking $f(z)$ be a uniform $U(a,b)$ pdf, we generate the following location family:
$$
g(x|\mu) = \frac{1}{b-a}, \quad a+\mu < x < b + \mu
$$
\vocab{Exponential location family.} By taking $f(z)$ to be Exponential($\beta$) pdf ($\beta$ is a fixed value), we generate the following location family
$$
g(x|\mu) = \frac{1}{\beta}e^{-(x-\mu)/\beta}, \quad \mu <x< \infty
$$

\section{September 27, 2021}
In this section we are going to consider events that \textit{co-occur}, and revisit concepts such as \textit{independence} and \textit{conditional probability}. We will learn how to handle random variables that co-occur.
\subsection{Joint and Marginal Distributions}
What is the most important information about a random variable? The PMF, or the PDF. The multivariate analogue to the \vocab{Joint function}, which takes in a value of two or more random variables, and returns the probability that those two variables jointly take on those values.
$$
\PP(X =x, Y=y) \quad \text{Joint Probability of $X$ and $Y$}
$$
Should be read a: ``Probability $X$ takes on the value $x$ and $Y$ takes on the value $y$".
\begin{definition}[Joint probability table]
 A joint probability table is a way of specifying the "joint" probability distribution between two random variables.    
\end{definition}
\section{September 29, 2021}
\subsection{Conditional Distributions and Independence}
\section{October 4, 2021}
\subsection{Bivariate Transformations}
In this lecture our focus will be on computing the pdf of a bivariate random vector $(U,V)$ defined by
$$
U = g_1(X,Y), \quad V = g_2(X,Y)
$$
where $(X,Y)$ is a random vector with a known joint pdf $f(x,y)$ and $g_1,g_2$ are defined functions.

\begin{theorem}
Let $(X,Y)$ be a bivariate random vector with joint pdf $f_{X,Y}(x,y)$ and $\SA = \{
(x,y): f_{X,Y}(x,y)>0
\}$.
Let $g=(g_1,g_2): \SA \to \SB$ be a one-to-one transformation. Denote $g^{-1}:=h = (h_1,h_2)$ and let $J$ be the Jacobian of the transformation
    $$
    J = 
    \begin{bmatrix} \frac{\partial h_1}{\partial u} &
    \frac{\partial h_1}{\partial v} \\ 
    \frac{\partial h_2}{\partial u} & 
    \frac{\partial h_2}{\partial v}
    \end{bmatrix}
    $$
  Then the joint pdf of $(U,V)$ is 
  $$
  f_{U,V}(u,v) = f_{X,Y}(h_1(u,v),h_2(u,v)) \cdot |J|, \quad (u,v) \in \SB.
  $$
\end{theorem}
If the transformation $g$ is not one-to-one on $\SA$, but we can find a partition $\SA_1,\SA_2,...,\SA_k$ such thart $g = g^{(i)}:\SA_i \to \SB_i$ is one-to-one for each $i = 1,...,k$, then we can still write down a formula for the joint pdf of $(U,V) = g(X,Y)$:
$$
f_{U,V}(u,v) = \sum_{i=1}^{j}f_{X,Y}(h_1^{(i)}(u,v),h_2^{(i)}(u,v)) \cdot |J_i|\II_{(u,v) \in \SB_i}
$$
where $h^{(i)} = (h_1^{(i)},h_2^{(i)})$ are the inverse of $g^{(i)}$ and $J_i$ si the corresponding Jacobian.
\section{October 6, 2021}
\subsection{Hierarchical Models}
Thus far we have seen random variables which have single distributions, possibly depending on parameters. However, we can think of the parameter of a distribution as being a random variable, which itself has a distribution. 

\begin{example}[Binomial-Poisson hierarchy]
Classic example of  hierarchical model is the following: An insect lays a large number of eggs, each surviving with probability $p$. On average, how many eggs will survive? 
The ``large number" of eggs laid is a random variable, often taken to be Poisson$(\lambda)$. Furthermore, if we assume that each eggs survival is independent, we have Bernoulli trials. Therefore if $X =$ number of survivors and $Y=$ number of eggs laid, we have
$$
X|Y \sim \text{binomial}(Y,p)
$$
$$
Y \sim \text{Poisson}(\lambda),
$$
a hierarchical model.
\end{example}
\term{Solution:}\\
The random variable of interest, $X =$ number of survivors, has the distribution given by
\begin{align*}
    \PP(X=x) &= \sum_{y=0}^{\infty} \PP(X=x,Y=y) \\
             &= \sum_{y=0}^{\infty} \PP(X=x|Y=y) \PP(Y=y) \\
             &= \sum_{y=x}^{\infty} \left[\binom{y}{x}p^x(1-p)^{y-x} \right] \left[ \frac{e^{-\lambda \lambda^y}}{y!} \right] \\
             & \vdots \quad \text{(after some algebraic simplifications)} \\ 
             &= \frac{(\lambda p)^x e^{- \lambda}}{x!} e^{(1-p) \lambda}
\end{align*}
Thus, $X \sim$Poission($\lambda p$). Thus, any marginal inference on $X$ is with respect to a Poisson($\lambda p$) distribution, with $Y$ playing no part at all. Introduction $Y$ in the hierarchy was mainly to aid our understanding the model. 
\\
Now we can easily compute the expected value
$$
\EE[X]= \lambda p
$$
so, on average, $\lambda p$ eggs will survive. 
\\
\\
Sometimes calculations can be greatly simplified using the following theorem. Recall that $\EE(X|y)$ is a function of $y$ and $\EE(X|Y)$ is a random variable whose distribution depends on the value of $Y$.

\begin{theorem}
    If $X$ and $Y$ are random variable, then
    $$\EE(X) = \EE \EE(X|Y)$$
    and,
    $$
    \VV ar(X) = \VV ar(\EE (X|Y)) + \EE (\VV ar(X|Y))
    $$
\end{theorem}
\begin{proof}
    By definition $$\underbrace{\EE(X|Y) = \sum_{x}x\PP(X=x|Y=y)}_\text{Average of $X$ when we fix $Y=y$}.$$ 
    But $Y$ is a random variable, so if average over all realizations of $Y$, we have,
    $$\EE_{Y}\left(\EE_X(X|Y)\right) = \sum_{y}\underbrace{ \sum_{x}x\PP(X=x|Y=y)}_{\EE(X|Y)}\cdot\PP(Y=y)$$
    by the definition of joint density, we can re-write the equation,
     $$
     \implies \sum_{y}\sum_{x}x\PP(x,y) = \sum_{x}\sum_{y}x\PP(x,y) = \sum_{x}x\underbrace{\sum_{y}\PP(x,y)}_{\PP(X=x)} = \sum_{x}x\PP(X=x) = \EE(X).
     $$
\end{proof}
\section{October 13, 2021}

\section{October 18, 2021}
\subsection{Multivariate Distributions}
The random vector $\boldsymbol{X} = (X_1,\dots,X_n)$ has a sample space in $\RR^n$. If $(X_1,\dots,X_n)$ is discrete random vector (the sample space is countable), then the \term{joint pmf} of $(X_1,\dots,X_n)$ is the function define by $f(\boldsymbol{x}) = f(x_1,\dots,x_n) = \PP(X_1 =x_1,\dots,X_n=x_n)$ for each $(x_1,\dots,x_n=x_1)$. Then for any $A \in \RR^n $,
$$
\PP(\boldsymbol{X} \in A) = \sum_{\boldsymbol{x} \in A} f(\boldsymbol{x})
$$
If $(X_1,\dots,X_n)$ is continuous random vector, the joint pdf of $(X_1,\dots,X_n)$ is a function $f(x_1,\dots,x_n)$ that satisfies 
$$
\PP(\boldsymbol{x}) = \int \dots \int_{A} f(\boldsymbol{x}) dx = \int \dots \int_{A} f(x_1 \dots x_n) dx_1 \dots dx_n
$$
These are $n-$ fold integrals with limits of integration set so that the integration is over all points of $ \boldsymbol{x} \in A$.

\begin{note}
\vocab{Marginal pdf or pmf} of any subset of of of the coordinates of $(X_1,\dots,X_n)$ can be computed by integrating or summing the joint pdf or pmf over all possible values of the other coordinates. 

$$
f(x_1,...,x_k) = \int_{- \infty}^{\infty} \dots \int_{- \infty}^{\infty} f(x_1,...,x_n)dx_{k+1}...dx_{n}
$$
\end{note}
\section{October 20, 2021}
\subsection{Basic Concepts of Random Samples}
\begin{definition}[Random sample]
The random variables $X_1,...,X_n$ are called a \vocab{random sample of size n} from population $f(x)$ if $X_1,...,X_n$ are mutually independent random variables and the marginal pdf or pmf of each $X_i$ is the same function $f(x)$
\end{definition}
The joint pdf of the random sample is given by
$$
f(x_1, \dots, x_n) = f(x_1) \cdots f(x_n)
= \prod_{i=1}^{n}f(x_i)
$$
The joint pdf can be used to calculate the probabilities involving samples. If the population pdf is a member of the parametric family with a pdf or pmf given by $f(x|\theta)$, then the joint pdf or pmf is
$$
f(x_1, \dots, x_n| \theta) = f(x_1| \theta) \cdots f(x_n| \theta)
= \prod_{i=1}^{n}f(x_i| \theta)
$$

\begin{definition}
    Let $X_1,...,X_n$ be a random sample and $T(x_1,...,x_n)$ be real vector valued function, whose domain includes the sample space of $X_1,...,X_n$. The random variable (or vector) $Y = T(X_1,...,X_n)$ is called a \vocab{statistic}.
\end{definition}

The definition of a statistics is very broad, with only one restriction being that a statistic cannot be a function of the parameter. The following three statistics are used to provide sample summaries for data :
\begin{enumerate}
    \item $\Bar{X} = \frac{X_1+ \cdots +X_n}{n}$, the sample mean.
    \item $S^2 = \frac{1}{n-1} \sum_{i=1}^{n}(X_i- \Bar{X})^2$, the sample variance
    \item $S^2 = \sqrt{S^2}$, standard deviation.
\end{enumerate}
The next result is useful for studying sampling distributions.
\begin{lemma}
    Let $X_1,...,X_n$ be a random sample from a population and let $g(x)$ be a function such that $\EE g(X_1)$ and $\VV ar [g(X_1)]$ exist. Then
    $$
    \EE \left(
    \sum_{i=1}^{n} g(X_i)
    \right)
    = n(\EE g(X_1))
    $$
    and
    $$
    \VV ar \left(
    \sum_{i=1}^{n} g(X_i)
    \right)
    = n(\VV ar g(X_1))
    $$
\end{lemma}
\begin{proof}
Details of the proof are in lecture notes and textbook.
\end{proof}
Let $X_2,...X_n$ be a random sampel from a population with mean $\mu$ and variance $\sigma^2 < \infty$. Then
\begin{enumerate}
    \item $\EE \Bar{X} = \mu$
    \item $\VV ar \Bar{X} = \frac{\sigma^2}{n}$
    \item $\EE S^2 = \sigma^2$
\end{enumerate}

\begin{theorem}
Let $X_1,...,X_n$ be a random sample from a population with a mgf $M_x(t)$. Then the mgf of $\Bar{X}$ is given by
$$
M_{\Bar{X}}(t) = \left[
M_{X}(\frac{t}{n})^n
\right]
$$
\end{theorem}
If the mgf does not exists or doesn't have a closed form, we can use the following result to derive the distribution of $\bar{X}$.

\begin{theorem}
    Let $X$ and $Y$ be independent continuous random variables with pdf's
    $f_X(x)$ and $f_Y(y)$. Then the pdf of $U = X+Y$ is given by
    $$
    f_U(u) = \int_{-\infty}^{\infty} f_{X}(x)f_{Y}(u-x)dx = 
    \int_{-\infty}^{\infty} f_{X}(u-y)f_{Y}(y)dy
    $$
\end{theorem}
\section{November 3, 2021}
\subsection{Sampling from Normal Distribution}
\begin{proposition}
    Let $X_1,...,X_n$ be a random sample from $N(\mu,\sigma^2)$ population, $\bar{X}$ be the sample mean and $S^2$ be the sample variance. Then, 
    \begin{enumerate}
        \item $\bar{X}$ and $S^2$ are \textbf{independent} random variables
        \item $\bar{X} \sim$ Normal($\mu,\sigma^2/n)$
        \item $(n-1)S^2/\sigma^2 \sim \chi_{n-1}^2$
    \end{enumerate}
\end{proposition}
Recall that if a random vector $(X,Y)$ has a normal distribution, then
$$
X,Y \quad \text{are independent } \iff \text{Cov}(X,Y) = 0
$$
We can actually generalize this idea to linear combinations of normal random variables. Let $X_1,...,X_n$ be random variables such that $X_i \sim N(\mu_i, \sigma_i^2)$ for each $i=1,..n$. Define the random vector $\boldsymbol{U} =(U_1,...,U_k)$ and $\boldsymbol{V} = (V_1,...,V_r)$ where
$$
U_i = \sum_{j=1}^n a_{ij}X_j, \quad i=1,...,k; \quad V_r = \sum_{j=1}^{n}b_{rj}X_j, \quad r=1,...,m
$$
and $ a_{ij}$, $b_{rj}$ are constants. Then,
$$
U_i \sim N \left( \sum_{j=1}^n a_{ij} \mu_i, \sum_{j=1}^n a_{ij}^2 \sigma_i^2 \right),
\quad
V_r \sim N \left( \sum_{j=1}^n b_{rj} \mu_i, \sum_{j=1}^n b_{rj}^2 \sigma_i^2 \right),
\quad
\text{Cov}(U_i,V_r) = \sum_{j=1}^n a_{ij} b_{rj} \sigma_j^2
$$
and, 
$$
U_i,V_r \text{ are independent } \iff \text{Cov}(U_i,V_r) = 0
$$
And this also implies that
$$
\boldsymbol{U},\boldsymbol{V} \text{ are independent random vectors } \iff U_i, V_r \text{ are independent } \forall i, \forall r.
$$
\vocab{Application:} If $X_1,...,X_n$ is a random sample from $N(\mu, \sigma^2)$ population, 
$$
X_j - \bar{X} \text{ and } \bar{X} \text{ are independent}
$$
for every $j=1,...,n$. From here we conclude that $S^2$ and $\bar{X}$ are independent. 

\subsection{The Derived Distributions: Student’s $t$ and Snedecor’s $F$}
\begin{definition}[Student's $t$ distribution]
We say that a random variable $T$ has \vocab{Student's t distribution} with $p$ degrees of freedom (and we write $T \sim t_p$) if its pdf is given by
$$
f(t) = \frac{\Gamma(p+1)/2}{\Gamma(p/2)} \cdot
\frac{1}{(1+t^2/p)^(p+1)/2}, \quad -\infty < t < \infty.
$$
\end{definition}
\textbf{Properties of the $t$ distribution}
\begin{enumerate}
    \item $t_1$ = Cauchy($0,1$)
    \item the graph of student's $t$ distribution is bell-shaped and symmetric around 0.
    \item We have
    $
    \EE T = 0, \quad \VV ar T = \frac{p}{p-2} \text{ if } p >2.
    $
\end{enumerate}
Let $U$ and $V$ be random variables such that $U \sim $ Normal(0,1) and $V \sim \chi_p^2$. Then
$$
T := \frac{U}{\sqrt{V/p}} \sim t_p.
$$
\vocab{Application.} Let $X_1,...,X_n$ be a random sample from Normal($\mu$, $\sigma^2$) population. Then
$$
T := \frac{\bar{X} - \mu}{S / \sqrt{n}} \sim t_{n-1}.
$$
\textbf{T is used to make inferences about $\mu$, when $\sigma^2$ is unknown.}
\begin{definition}[Snedecor’s F distribution]
We say that a continuous random variable $F$ has a Snedecor’s F distribution with $p$ and $q$ degrees of freedom (and we write $F \sim F_{p,q}$) if its pdf is given by
$$
f(x) = \frac{\Gamma((p+q)/2)}{\Gamma(p/2)\Gamma(q/2)} \cdot \left( 
\frac{p}{q}
\right)^{p/2} \cdot
\frac{x^{(p/2)-1}}{[1+(p/q)x]^{(p+q)/2)}}, \quad 0 < x < \infty.
$$
\end{definition}

\section{November 8, 2021}
\subsection{Order Statistics}
We will consider a transformation that takes  $n$ RV's $X_1,...,X_n$ and essentially returns them in a sorted order. 
\begin{definition}
The \vocab{order statistics} of random variables $X_1,...,X_n$ are the random variables $X_{(1)},...X_{(n)}$, where
\begin{align*}
X_{1} & = min(X_{(1)},...X_{(n)})\\
X_{2} & = \text{ is the second-smallest of } X_1,...,X_n \\
& \vdots \\
X_{n-1} & = \text{ is the second-largest of } X_1,...,X_n \\
X_{n} & = max(X_{(1)},...X_{(n)}).
\end{align*}
\end{definition}
The sample range is a statistic defined as $R = X_{(n)} - X_{(1)}$. The midrange statistic is defined as $V = (X_{(1)} + X_{(n)})/2$. The \textit{sample median} is defined by
$$
M = 
\begin{cases}
    X_{\frac{n+1}{2}} \quad \text{if } n \text{ if odd} \\
   \frac{1}{2}(X_{\frac{n}{2}} + X_{\frac{n}{2} + 1} ) \quad \text{if } n \text{ if odd}
\end{cases}
$$
Its important to note that order statistics $X_{(1)},...X_{(n)}$ are random variables, and each $X_{(i)}$ is a function of the random sample $X_1,...,X_n$. Even if the original sample is independent, the order statistics are \textbf{dependent}! 
\begin{theorem}
Let $X_1,...,X_n$ be a random sample from discrete distribution with pmf $f(x)$. Suppose that the sample space of $X_1$ is a set $\{x_1,...,x_n \}$ such that $x_1 < x_2 < ...$ with CDF $F(x)$. Then the CDF of the $j^{th}$ order statistic $X_{(j)}$ is. 
$$
\PP(X_{(j)} \leq x) = \sum_{k=j}^{n} {x \choose n} F(x) ^{k}(1 - F(x))^{n-k}
$$
The pdf of the $j^{th}$ order statistic $X_{(j)}$ is given by 
$$
f(X_{j}(x)) = \frac{n!}{(j-1)!(n-1)!}f(x)[F(x)]^{j-1}[1-F(x)]^{n-1}
$$
\end{theorem}
Let $X_{(1)},...,X_{(n)}$ denote the order statistics of a random sample, $X_1,...,X_n$, from continuous population with cdf $F_X(x)$ and pdf $f_X(x)$. Then the joint pdf of $X_{(i)}$ and $X_{(j)}$, $1 \leq i < j \leq n$, is 
\begin{align*}
    f_{X_{(i)},X_{(j)}}(u,v) = \frac{n!}{(i-1)!(j-1-i)! (n-j)!}
f_{X}(u)f_X(v)[F_X(u)]^{i-1} \\
\times [F_X(u) - F_X(u)]^{j-1-i}[1-F_X(v)]^{n-j}
\end{align*}
 


\section{November 10, 2021}
\subsection{Convergence Concepts}
The notion of letting sample size approach infinity can provide us with useful approximations, since it usually happens that expressions become simplified in the limit. 
\\
\begin{definition}
    A sequence of random variables, $X_1,X_2,...$ converges in probability to a random variable $X$ if, for every $\eps > 0$,
    $$
    \lim_{n \to \infty}\PP(|X_n -X| \geq \eps) = 0  \quad \text{or equivalently,} \quad  \lim_{n \to \infty}\PP(|X_n -X| < \eps) = 1
    $$
\end{definition}
\section{November 15, 2021}
\subsection{Inference}
Our objective is now to use the information in the sample $X_1,...,X_n$ to make inferences about the unknown parameter.
\\
\\ 
For $j = 1,..m$, let $T_j$ be measurable functions defined on $\RR^n \to \RR$ and not depending on $\theta$, and let $\boldsymbol{T} = (T_1,...,T_m)'$. Then
$$
\boldsymbol{T}(X_1,...,X_n) = \left(
T_1(X_1,...,X_n),...,T_m(X_1,...X_n)
\right)'
$$
is called a \vocab{m-dimensional statistic}.
Any statistic T($\boldsymbol{X}$) defines a form of data reduction by partitioning the sample space $\SX$. If we only use the value of the statistic, T($\boldsymbol{x}$), rather than the entire sample $\boldsymbol{x}$, the two samples $\boldsymbol{x}$ and $\boldsymbol{y}$ will be treated equally if T($\boldsymbol{x}) =$ T($\boldsymbol{y}$) 
\subsection{Sufficiency Principle}
\begin{definition}
    Let $\boldsymbol{X} = (X_1,...,X_n)$ be a random sample with a join pdf or pmf denoted by $f(\boldsymbol{x}|\theta)$. Let T = T($\boldsymbol{x}$)
    be a statistic based on this sample with pdf (or pmf) denoted by $f_T(t|\theta)$. We say that T is a \vocab{sufficient statistic} for $\theta$ if for any $t$ such that $f_T(t|\theta) > 0$, the conditional pdf (or pmf) of $\boldsymbol{X}$ given $T=t$ does not depend on $\theta$.
    \end{definition}
\term{Remark.} The joint pdf (or pmf) of $(\boldsymbol{X},T)$ is 
    $$
    f_{X,T}(\boldsymbol{x},t| \theta) =
        \begin{cases}
            f(\boldsymbol{x}| \theta) \quad \text{if } t= T(\boldsymbol{x}) \\
            0 \quad \text{otherwise}
        \end{cases}
    $$
\section{November 17, 2021}
\subsection{Factorization Theorem Continued}
\begin{example}
    Let $X_1,...,X_n$ be independent random variables such that $$X_k \sim Unif(k(\theta -1), k(\theta+1)).$$
    
    Show that $ \left ( \displaystyle \underset{1 \leq k \leq n}{\mathrm{min}} \frac{x_k}{k}, \underset{1 \leq k \leq n}{\mathrm{max}} \frac{x_k}{k}
    \right )$ is a $2-$dimensional sufficient statistic. 
    
\end{example}

\begin{theorem}
    Let $\boldsymbol{X} = (X_1,...,X_j)$ be a randome sample from a pdf(or pmf) which belongs to the following exponential family
    $$
    f_X(x|\theta) = h(x)c(\theta)exp 
    \left \{ 
    \sum_{i=1}^{k}w_i(\theta)t_i(x)
    \right \}
    $$
    with $\theta = (\theta_1,...,\theta_d)$. Suppose that $d \leq k$. Define 
    $$T_i = T_i(\boldsymbol{X}) = \sum_{j=1}^{n}T_i(X_j)
    \quad \forall i=1,...,k.$$
    Then $\boldsymbol{T} = (T_1,...,T_k)$ is a sufficient statistic for $\theta$.
\end{theorem}
\begin{proof}
    The joint pdf of $\boldsymbol{X}$ is:
    \begin{align*}
    f(\boldsymbol{x}|\theta) & = 
    \prod_{j=1}^{n}f_{x_j}(x_j \theta) 
    = \prod_{j=1}^{n} \left[
    h(x_j)c(\theta) \left \{
    \sum_{i=1}^{k}w_i(\theta)t_i(x_j)
    \right\}
    \right ] \\
    & = \prod_{j=1}^{n}h(x_j)[c(\theta)]^n \mathrm{exp} \left \{
    \sum_{i=1}^{k}w_i(\theta)t_i(x_j)
    \right\}
    \end{align*}
    Clearly, 
    $$
    \underbrace{ \prod_{j=1}^{n}h(x_j)}_{h'(x)}
    \underbrace{[c(\theta)]^n\mathrm{exp} \left \{
    \sum_{i=1}^{k}w_i(\theta)t_i(x_j)
    \right\}}_{g(T_1(\boldsymbol{x}),...,T_k(\boldsymbol{x})|\theta)}
    $$
    Thus, by the Factorization theorem $(T_1(\boldsymbol{x}),...,T_k(\boldsymbol{x})|\theta)$ is a sufficient statistic for $\theta$.
\end{proof}
\term{Remark:} If $\boldsymbol{T}$ is a sufficient statistic for $\theta$, then any one-to-one transformation of $\boldsymbol{S} = \pi(\boldsymbol{T})$ is also a sufficient statistic for for $\theta$, $\forall \pi$ one-to-one maps. \textbf{Therefore, a sufficient statistic is not unique}. 
\begin{definition}
    A sufficient statistic is $\boldsymbol{T} = (T_1,...,T_k)$ is called a \vocab{minimal sufficient statistic} for $\theta$ if, for any other sufficient statistic $\boldsymbol{T}^* = (T_1^*,...,T_k^*)$, there exists a function $\phi$ such that $\boldsymbol{T} = \phi(\boldsymbol{T}^*)$. This is equivalent to saying that
    $$
    T^*(\boldsymbol{x}) = T^*(\boldsymbol{y}) \Longrightarrow T(\boldsymbol{x}) = T(\boldsymbol{y})
    $$
\end{definition}
\term{Note:} A minimal sufficient statistic may \textbf{NOT} be unique.
\section{November 22, 2021}
\section{November 24, 2021}
\subsection{Equivariance Principle}
\textbf{Equivariance Principle:} If $\boldsymbol{Y} = g(\boldsymbol{X})$ is a change of measurement scale such that the model for $\boldsymbol{Y}$ has the same formal structure as the model for $\boldsymbol{X}$,  then inference procedure should be both measurement equivariant and formally equivariant. 
\begin{definition}
    A set $\SG$
\end{definition}
\section{November 29, 2021}
\subsection{Invariance property of MLE}
Suppose that the distribution is index by the parameter $\theta$, if are interested in estimating some function of $\theta$, say $\tau(\theta)$.
If we let $\eta = \tau(\theta)$, then the inverse function $\theta = \tau^{-1}(\eta)$ is well-defined, and we can express the likelihood function as a function of $\eta$
$$
L^{*}(\eta|\boldsymbol{x}) \prod_{i=1}^{n}f(x_i|\tau^{-1}(\eta)) = L(\tau^{-1}(\eta)|\boldsymbol{x})
$$
and,
$$
\underset{\eta}{\text{sup}}L^{*}(\eta|\boldsymbol{x}) = \underset{\eta}{\text{sup}}L(\tau^{-1}(\eta)|\boldsymbol{x})
= \underset{\eta}{\text{sup}}L(\theta|\boldsymbol{x})
$$ 
The \vocab{induced likelihood} of $\eta$ is defined by 
$$
\underset{\eta}{\text{sup}}L^{*}(\eta|\boldsymbol{x}) =
\underset{\theta: \tau(\theta) = \eta}{\text{sup}}L(\theta|\boldsymbol{x})
$$
The value of $\Hat{\eta}$ which maximized $L^{*}(\eta|\boldsymbol{x})$ is called the MLE of $\eta$:
$$
\underset{\eta}{\text{sup}}L^{*}(\eta|\boldsymbol{x}) =L^{*}(\Hat{\eta}|\boldsymbol{x})
$$
\section{December 1, 2021}
\subsection{Methods of evaluating estimators}
\subsection{Best unbiased estimators}
The general topic of evaluating statistical procedures is part of a branch of statistics called decision theory. In this section we look at some basic criteria for evaluating estimators.
    
\begin{definition}[Mean square error (MSE)]
    The \vocab{mean square error} of an estimator $W$ of a parameter $\theta$ is the function of $\theta$ defined by $\EE_{\theta}(W-\theta)^2$.
\end{definition}

\begin{definition}[Bias]
    The bias of an estimator $T$ for a parameter $\theta$ is defined by 
    $$
    \text{Bias}_{\theta}(T) = \EE_{\theta}T -\theta
    $$
\end{definition}
\textbf{Note:} $MSE$ incorporates two components, one measuring the variability and the other measuring bias.
$$
\text{MSE}_{\theta} = \VV ar_{\theta}(T) + (\text{Bias}_{\theta}(T))^2
$$
\begin{example}
    Let $X-1,...,X_n$ be iid $n(\mu,\sigma^2)$. The statistics $\bar{X}$ and $\S^2$ are both biased estimators since
    $$
    \EE \bar{X} = \mu \quad, \EE S^2 = \sigma^2, \quad \text{for all $\mu$ and $\sigma$}.
    $$
    The MSEs of the estimators are given by
    $$
    \EE(\bar{X} - \mu)^2 = \VV ar \bar{X} = \frac{\sigma}{n}.
    $$
    $$
    \EE(S^2 \sigma^2)^2 = \VV ar S^2
    $$
\end{example}
\section{December 6, 2021}
\subsection{methods of evaluating estimators (continued)}

\subsection{Sufficiency and Unbiasedness}
\section{December 8}








\end{document}
