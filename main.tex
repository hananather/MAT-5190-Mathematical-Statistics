
\documentclass[8.5pt,twoside=semi,openright,numbers=noenddot]{scrartcl}

\usepackage{amsmath,amssymb,amsthm}
\usepackage[usenames,svgnames,dvipsnames]{xcolor}

\usepackage{amsmath,amssymb,amsthm}
\usepackage{mathrsfs}
\usepackage[usenames,svgnames,dvipsnames]{xcolor}
\usepackage{hyperref}
\usepackage[nameinlink]{cleveref}
\usepackage{textcomp}
\usepackage{enumerate}
\usepackage{mathtools}
\usepackage{microtype}
\usepackage{geometry}
\usepackage[normalem]{ulem}
\usepackage{stmaryrd}
\usepackage{wasysym}
\usepackage{multirow}
\usepackage{prerex}
\usepackage{marginnote}
\usepackage{sidenotes}
\usepackage{mathtools}


\usepackage[headsepline ]{scrlayer-scrpage}
\renewcommand{\headfont}{}
\addtolength{\textheight}{3.14cm}
\setlength{\footskip}{0.5in}
\setlength{\headsep}{10pt}


\automark[section]{section}


\rohead{\footnotesize\thepage}
\rehead{\footnotesize \textbf{\sffamily MAT 5190}, Hanan Ather}
% More script letters etc.
\newcommand{\SA}{\mathcal A}
\newcommand{\SB}{\mathcal B}
\newcommand{\SC}{\mathcal C}
\newcommand{\SD}{\mathcal D}
\newcommand{\SE}{\mathcal E}
\newcommand{\SF}{\mathcal F}
\newcommand{\SH}{\mathcal H}
\newcommand{\SX}{\mathcal X}
\newcommand{\SY}{\mathcal Y}
\newcommand{\iid}{\text{ i.i.d.}}


\hypersetup{
    colorlinks,
    linkcolor={red!50!black},
    citecolor={green!50!black},
    urlcolor={blue!80!black}
}

\renewcommand*{\marginfont}{\color{red}\sffamily}

\newcommand{\II}{\mathbb I}
\newcommand{\CC}{\mathbb C}
\newcommand{\EE}{\mathbb E}
\newcommand{\FF}{\mathbb F}
\newcommand{\NN}{\mathbb N}
\newcommand{\QQ}{\mathbb Q}
\newcommand{\RR}{\mathbb R}
\newcommand{\PP}{\mathbb P}
\newcommand{\VV}{\mathbb V}

\newcommand{\ZZ}{\mathbb Z}



\newcommand{\eps}{\varepsilon}
\newcommand{\abs}[1]{\left\lvert #1 \right\rvert}
\newcommand{\norm}[1]{\left\lVert #1 \right\rVert}
\newcommand{\dang}{\measuredangle} %% Directed angle
\newcommand{\ray}[1]{\overrightarrow{#1}} 
\newcommand{\seg}[1]{\overline{#1}}
\newcommand{\arc}[1]{\wideparen{#1}}


\newcommand{\vocab}[1]{\textbf{\color{blue} #1}}
\newcommand{\term}[1]{\textit{\color{red} #1}}


\usepackage{thmtools}
\usepackage[framemethod=TikZ]{mdframed}
\usepackage{hyperref}
\usepackage[nameinlink]{cleveref}


\usepackage[headsepline]{scrlayer-scrpage}
\definecolor{crimsonglory}{rgb}{0.75, 0.0, 0.2}
\definecolor{denim}{rgb}{0.08, 0.38, 0.74}
\definecolor{egyptianblue}{rgb}{0.06, 0.2, 0.65}
\definecolor{lava}{rgb}{0.81, 0.06, 0.13}

\renewcommand*{\sectionformat}{\color{lava}\S\thesection\autodot\enskip}
\renewcommand*{\subsectionformat}{\color{Blue} \S\thesubsection\autodot\enskip}


\addtokomafont{partprefix}{\rmfamily}



\theoremstyle{definition}
\mdfdefinestyle{mdbluebox}{%
	roundcorner = 10pt,
	linewidth=1pt,
	skipabove=12pt,
	innerbottommargin=9pt,
	skipbelow=2pt,
	nobreak=true,
	linecolor=blue,
	backgroundcolor=TealBlue!5,
}
\declaretheoremstyle[
	headfont=\sffamily\bfseries\color{MidnightBlue},
	mdframed={style=mdbluebox},
	headpunct={\\[3pt]},
	postheadspace={0pt}
]{thmbluebox}


\mdfdefinestyle{mdredbox}{%
	linewidth=0.5pt,
	skipabove=12pt,
	frametitleaboveskip=5pt,
	frametitlebelowskip=0pt,
	skipbelow=2pt,
	frametitlefont=\bfseries,
	innertopmargin=4pt,
	innerbottommargin=8pt,
	nobreak=true,
	linecolor=RawSienna,
	backgroundcolor=Salmon!5,
}
\declaretheoremstyle[
	headfont=\bfseries\color{RawSienna},
	mdframed={style=mdredbox},
	headpunct={\\[3pt]},
	postheadspace={0pt},
]{thmredbox}

\mdfdefinestyle{mdgreenbox}{%
	skipabove=8pt,
	linewidth=2pt,
	rightline=false,
	leftline=true,
	topline=false,
	bottomline=false,
	linecolor=ForestGreen,
	backgroundcolor=ForestGreen!5,
}
\declaretheoremstyle[
	headfont=\bfseries\sffamily\color{ForestGreen!70!black},
	bodyfont=\normalfont,
	spaceabove=2pt,
	spacebelow=1pt,
	mdframed={style=mdgreenbox},
	headpunct={ --- },
]{thmgreenbox}
\declaretheoremstyle[
	headfont=\bfseries\sffamily\color{ForestGreen!70!black},
	bodyfont=\normalfont,
	spaceabove=2pt,
	spacebelow=1pt,
	mdframed={style=mdgreenbox},
	headpunct={},
]{thmgreenbox*}

\mdfdefinestyle{mdblackbox}{%
	skipabove=8pt,
	linewidth=3pt,
	rightline=false,
	leftline=true,
	topline=false,
	bottomline=false,
	linecolor=black,
	backgroundcolor=RedViolet!5!gray!5,
}
\declaretheoremstyle[
	headfont=\bfseries,
	bodyfont=\normalfont,
	spaceabove=2pt,
	spacebelow=1pt,
	mdframed={style=mdblackbox},
	headpunct={ --- }
]{thmblackbox}

\mdfdefinestyle{mdbluebox*}{%
	skipabove=8pt,
	linewidth=1.5pt,
	rightline=false,
	leftline=true,
	topline=false,
	bottomline=false,
	linecolor=Blue!70!black,
	backgroundcolor=TealBlue!5,
}
\declaretheoremstyle[
	headfont=\bfseries\sffamily\color{Blue!70!black},
	bodyfont=\normalfont,
	spaceabove=2pt,
	spacebelow=1pt,
	mdframed={style=mdbluebox*},
	headpunct={},
]{thmbluebox*}

\newenvironment{note}{%
	\begin{mdframed}[linecolor=Blue,backgroundcolor=TealBlue!5, roundcorner = 10pt, nobreak=true]%
	\color{black}}%
	{\end{mdframed}}






\declaretheorem[%
style=thmgreenbox*,name=Theorem,numberwithin=section]{theorem}
\declaretheorem[style=thmblackbox,name=Lemma,sibling=theorem]{lemma}
\declaretheorem[style=thmbluebox,name=Corollary,sibling=theorem]{corollary}
\declaretheorem[style=thmblackbox,name=Example,sibling=theorem]{example}
\declaretheorem[style=thmbluebox*,name=Definition,sibling=theorem]{definition}
\declaretheorem[%
style=thmbluebox*,name=Fact,numberwithin=section]{proposition}

\newenvironment{moral}{%
	\begin{mdframed}[linecolor=green!70!black]%
	\bfseries\color{green!50!black}}%
	{\end{mdframed}}

\newenvironment{fact}{%
	\begin{mdframed}[linecolor=black]}%
	{\end{mdframed}}

\newenvironment{question}{%
	\begin{mdframed}[linecolor=red!70!black]%
	\bfseries\color{RawSienna}%
	\color{black} \textbf{Question:}} 
	{\end{mdframed}}


\declaretheoremstyle[
	headfont=\bfseries\color{RawSienna},
	mdframed={style=mdredbox},
	headpunct={\\[3pt]},
	postheadspace={0pt},
]{thmredbox}

\title{Mathematical Statistics I}
\subtitle{MAT 5190}

\author{Hanan Ather}
\date{Fall 2021}





\begin{document}
\maketitle
\tableofcontents
\newpage

\begin{abstract}
    These are course notes for MAT 5190.
\end{abstract}




\section{September 8, 2021}
\begin{definition}
    A \vocab{probability function} denoted by \PP is a (set) function which assigns each event $A$ a number denoted by $\PP(A)$, called the probability of A, and satisfies the following requirements:
    
\end{definition}
\subsection{Bayes' Rule}
\subsection{Random Variables}
\subsection{Transformation of expectations}
\section{September 13, 2021}
\subsection{Product Probability Spaces}
If we consider experiments $\SE_1$ and $\SE_2$ with respective probability spaces $(S_1,\SF_1, \PP_1)$ and $(S_2,\SF_2, \PP_2)$, then ($\SE_1, \SE_2$) = $\SE_1\ \times \SE_2$ has the sample space $S = S_1 \times S_2$. The appropriate $\sigma$-field $\SF$ of events in $S$ is defined by first defining the class $\SC$:
$$
\SC = \{ \SF_1 \times \SF_2: F_1 \in \SF_1, F_2 \in \SF_2 \}
$$
$$
\text{where, } \SF_1 \times \SF_2 = \{ (s_1,s_2): s_1 \in A_1, s_2 \in A_2 \}
$$
\subsection{Discrete Distributions}


\section{September 15, 2021}
\subsection{Continuous Distributions}
A random variable $X$ has a \textit{continuous distribution} if its cdf is continuous. Today's lecture covered some of the most commonly used continuous distributions. \\ \\
\vocab{Uniform Distribution}. This is a continuous distribution with pdf:
$$
f(x|a,b) = \frac{1}{b-a}, \quad a \leq x \leq b, \quad \text{where }, -\infty < a < b < \infty 
$$
\vocab{Gamma Distribution}.  This is a continuous distribution with pdf:
$$
f(x|\alpha, \beta) = \frac{1}{\Gamma(\alpha)\beta^{\alpha}}x^{\alpha -1}e^{-x/\beta}, \quad, 0< x < \infty.
$$
where $ 0 < \alpha$, $\beta < \infty$ and $\Gamma(\alpha) = \int_{0}^{\infty}x^{\alpha -1}e^{-x} dx$ is the \term{gamma function}. We write $X \sim$Gamma$(\alpha,\beta)$. Note that 
$$
\Gamma(\alpha +1) = \alpha\Gamma(\alpha),\quad \Gamma(1) =1, \quad \Gamma \left( \frac{1}{2} \right) =\sqrt{\pi}
$$
\vocab{Beta Distribution}. This is a continuous distribution with pdf:
$$
f(x|\alpha, \beta) = \frac{1}{B(\alpha, \beta)}x^{\alpha -1}(1-x)^{\beta -1}, \quad 0 < x <1, \text{where } 0 < \alpha, \beta < \infty.
$$
Note that $B(\alpha, \beta) = \int_{0}^{1}x^{\alpha -1}(1-x)^{\beta -1} dx$  is the \term{beta function.} We write $X \sim$Beta($\alpha, \beta$). We have
$$
B(\alpha, \beta) = \cfrac{\Gamma(\alpha)\Gamma(\beta)}{\Gamma(\alpha + \beta)}
$$
\vocab{Cauchy Distribution}. This is a continuous distribution with the pdf:
$$
f(x|\theta) = \frac{1}{\pi} \cdot \ \frac{1}{1 + (x- \theta)^2}, \quad -\infty < x< \infty
$$
Where $-\infty < \theta < \infty$. This is a ``pathological" example, since $\EE X = \infty$.\\
\vocab{Lognormal Distribution}. This is a continuous distribution with the following pdf:
$$
f(x|\mu, \sigma^2) = \frac{1}{\sqrt{2\pi} \sigma} \cdot \frac{1}{x}
exp \left \{
-\frac{(logx - \mu)^2}{2\sigma}\right \}, \quad 0 < x < \infty
$$
where $-\infty < \mu < \infty$, $0 < \sigma < \infty$. We write $X \sim$ Lognormal$(\mu, \sigma^2)$.
\\
\vocab{Double Exponential Distribution}. This is a continuous distribution with pdf:
$$
f(x|\mu,\sigma) = \frac{1}{2\sigma}exp \left \{
-\frac{\abs{x-\mu}}{\sigma}, \quad -\infty<x<\infty
\right \}
$$
Where $-\infty< \mu \infty$, $0 < \sigma < \infty$. We write $X \sim $Double Exponential($\mu,\sigma$). Its pretty straight forward to check that
$$
\EE X = \mu, \quad \VV(X) = 2 \sigma^2.
$$

\section{September 20, 2021}
\subsection{Exponential Family}
A family of pdf's (or pmf's) is called \textit{exponential family} if it can be written in the form
\begin{align}\label{exp_family}
    f(x|\theta) = h(x)c(\theta)\text{exp} \left \{ \sum_{i=1}^kt_i(x)w_i(\theta)\right \}
\end{align}
where $h(x) \geq 0, c(\theta) \geq 0, w_i(\theta) \in \RR, t_i(x) \in \RR$. Here the parameter $\theta = {\theta_1,...,\theta_d}$ is vector-valued. If $d=k$, the family is called a \textit{full exponential family}. If $d < k$, the family is called a \textit{curved exponential family}. 

\begin{example}[Binomial family with $n$ known]
    $$f(x|p) = {n \choose x} p^{x}(1-p)^{n} \left ( \frac{p}{1-p}\right )^x = 
    {n \choose x} (1-p)^{n}\text{exp} \left \{ xlog \left( \frac{p}{1-p}\right) \right \}$$
    $$
    =h(x)c(\theta)\text{exp} \left \{ \sum_{i=1}^kt_i(x)w_i(\theta)\right \}
    $$
\end{example}
\begin{example}[Poisson Family]
    $$
    f(x|\lambda) = \frac{e^{-\lambda}\lambda^{x}}{x!}
    =\frac{1}{x!}e^{\lambda}\text{exp}\{xlog\lambda\} =
    h(x)c(\theta)\text{exp} \left \{ \sum_{i=1}^kt_i(x)w_i(\theta)\right \}
    $$
\end{example}

\begin{theorem}
If $X$ is a random variable whose probability density function is given by \ref{exp_family} then
$$
\EE \left (
\sum_{i=1}^{k} = \frac{\partial w_i(\theta)}{\partial\theta_j}t_i(X)\right)
= -\frac{\partial}{\partial_j} \text{log} c(\theta)
$$

$$
\VV ar \left (
\sum_{i=1}^{k}  \frac{\partial w_i(\theta)}{\partial\theta_j}t_i(X)\right)
= -\frac{\partial^2}{\partial \theta_j} \text{log} c(\theta) 
- \EE \left (
\sum_{i=1}^{k} = \frac{\partial w_i(\theta)}{\partial^2 \theta_j}t_i(X)\right)
= -\frac{\partial^2}{\partial \theta_j^2} \text{log} c(\theta)$$
\end{theorem}
\subsection{Natural Parameterization}
If we use $\eta_i = w_i(\theta)$ in formula (\ref{exp_family}) and $\eta =(\eta_1,...,\eta_k)$, we obtain the \textit{natural parametrization}:

\begin{align}\label{natural}
    f(x|\eta) = h(x)c^{*}(\eta) \text{exp} \left \{ 
    \sum_{i =1}^{k}t_i(x) \eta_i
    \right \}
\end{align}
where $h(x)$ and $t_i(x)$ are the same as in formula (\ref{natural}). The natural space is $\SH = \{\eta: \int_{-\infty}^{\infty}\text{exp}\{ \sum_{i =1}^{k}t_i(x) \eta_i\}$. We have
$$
c^*(\eta) = \frac{1}{\int_{-\infty}^{\infty}\text{exp}\{ \sum_{i =1}^{k}t_i(x) \eta_i\}}, \quad \eta \in \SH
$$
\vocab{Normal Family.} The natural parametrization if $\eta_1 = 1/\sigma^2$, $\eta_2 = \mu/\sigma^2$ with natural parameter space $\SH = \{(\eta_1,\eta_2): 0 < \eta_1 < \infty, -\infty < \eta_2 < \infty \}$. We have $\mu = \eta_2/ \eta_1$ and $\sigma^2 = 1/\eta_1$ and hence the normal pdf 
$$
f(x|\mu, \sigma) = \frac{1}{\sqrt{2\pi}\sigma}\text{exp}\left \{ 
-\frac{\mu^2}{2\sigma^2}
\right \}
\text{exp}\left \{ 
-\frac{x^2}{2\sigma^2} + \frac{x \mu}{ \sigma^2}
\right \}
$$
can be written as,
$$
f(x|\eta_1, \eta_2) = \frac{\sqrt{\eta_1}}{\sqrt{2\pi}}
\text{exp}\left \{ 
-\frac{\eta_2^2}{2\eta_1}
\right \}
\text{exp}\left \{ 
-\frac{\eta_1x^2}{2} + \eta_2x
\right \}
$$
$$
= f(x|\eta) = h(x)c^{*}(\eta) \text{exp} \left \{ 
    \sum_{i =1}^{k}t_i(x) \eta_i
    \right \}
$$

\section{September 22, 2021}
\subsection{Location and Scale Families}
\section{September 27, 2021}
In this section we are going to consider events that \textit{co-occur}, and revisit concepts such as \textit{independence} and \textit{conditional probability}. We will learn how to handle random variables that co-occur.
\subsection{Discrete Joint and Marginal Distributions}

\begin{definition}[Joint PMF]
    Let $(X,Y)$ be a bi-variate random vector. We day that the distribution of $(X,Y)$ is \textit{discrete} if the possible values of $(X,Y)$ are countable. In this case the function,
    $$
    f(x,y) = f_{X,Y}(x,y) = \PP(X=x,Y=y)
    $$
 is called the \vocab{joint pmf} of $X,Y$.
\end{definition}

What is the most important information about a random variable? The PMF, or the PDF. The multivariate analogue to the \vocab{Joint function}, which takes in a value of two or more random variables, and returns the probability that those two variables jointly take on those values.
$$
\PP(X =x, Y=y) \quad \text{Joint Probability of $X$ and $Y$}
$$
Should be read a: ``Probability $X$ takes on the value $x$ and $Y$ takes on the value $y$".

 A joint probability table is a way of specifying the "joint" probability distribution between multiple random variables. It does to by keeping a multi-dimensional lookup table, so essentially any assignment of the random variables, $\PP(X=x,Y=y,...)$ can be directly looked up. A probability mass table is a brute force way to store the joint probabilities of random variables. 
 
 \begin{note}
    \vocab{Property 1}. If $A$ is a subset of $\RR^2$
    $$
    \PP((X,Y) \in A) = \sum_{(x,y) \in A}\PP(X=x,Y=y)
    $$
    \\
    \vocab{Property 2.} If $g(x,y)$ is a real-valued function, then
    $$
    \EE g(X,Y) = \sum_{x,y} g(x,y)f(x,y)
    $$
 \end{note}

\begin{definition}[Marginal pmf] 
    Let $f(x,y)$ be the joint PMF of the discrete random vector $(X,Y)$, the PMF of $X$ is called the \vocab{marginal pmf} of $X$, denoted by $f_X(x)$. Similarly,  the PMF of $Y$ is called the marginal PMF of $Y$, denoted by $f_Y(y)$.
\end{definition}
The marginals can be computed by the following formulas:
$$
f_X(x) = \sum_{y \in \RR} f_{X,Y}(x,y), \quad  f_Y(y) = \sum_{x \in \RR} f_{X,Y}(x,y)
$$

\subsection{Continuous Joint and Marginal Distributions}
\begin{definition}
    Let $(X,Y)$ be a bivariate random vector. We say the distribution of $(X,Y)$ is \textit{continuous} if the joint CDF of $(X,Y)$ is defined by $F(u, v) = \PP(X \leq u, Y \leq v)$ is continuous. In this case the function $f(x,y)$ which satisfies the condition
    $$
    F(u, v) = \int_{ -\infty}^{\infty} \int_{-\infty}^{\infty} f(x,y)dxdy
    $$
    is called the \vocab{joint pdf} of $(X,Y)$.
    \\
    Note that 
    $$
    \int_{ -\infty}^{\infty} \int_{-\infty}^{\infty} f(x,y)dxdy = 1, \quad
    f(x,y) \geq 0 \text{ for all } x,y
    $$
\end{definition}

\subsection{Multinomial}
The multi-nomial distribution is a \vocab{parametric} distribution for multiple random variables. 

\section{September 29, 2021}
\subsection{Conditional Distributions (Discrete)}
Let $(X,Y)$ be discrete bivariate random vector with joint pmf $f(x,y)$ and marginal pmf's $f_X(x)$ and $f_Y(y)$. For all fixed $x \in \SX = 
\{
x:f_X(x) > 0
\}$, we define
$$
f(y|x) = \frac{f(x,y)}{f_X(x)}
$$
The function $y \mapsto f(y|x)$ is called \term{conditional pmf} of $Y$ given $X=x$.
For every $x \in \SX$, the function $y \mapsto f(y|x)$ is a pmf, since
$$
f(y|x) \geq 0 \text{ for all } y, \quad \text{and} \quad \sum_{y}f(y|x) =1.
$$

\begin{note}
    \vocab{Property 1.}
    For every fixed $x \in \SX$ and for every set $A$
    $$
    \PP(Y \in A|X=x) = \sum_{y \in A}f(y|x). 
    $$
    The function $y \mapsto F(y|x):= \PP(Y \leq y|X=x)$ is called the \term{conditional cdf} of $Y$ given that $X=x$.
    \\
    \vocab{Property 2.} For every fixed $x \in \SX$ and for every real-valued function $g$
    $$
    \EE (g(Y)|X=x) = \sum_y g(y)f(y|x).
    $$
\end{note}
\subsection{Conditional Distributions (Continuous)}
Let $(X,Y)$ be continuous random vector with joint pdf $f(x,y)$ and marginal pdf's $f_X(x)$ and $f_Y(y)$. For all fixed $x \in \SX = 
\{
x:f_X(x) > 0
\}$, we define
$$
f(y|x) = \frac{f(x,y)}{f_X(x)}
$$
The function $y \mapsto f(y|x)$ is called \term{conditional pdf} of $Y$ given $X=x$.
For every $x \in \SX$, the function $y \mapsto f(y|x)$ is a pdf, since
$$
f(y|x) \geq 0 \text{ for all } y, \quad \text{and} \quad \int_{-\infty}^{\infty}f(y|x) =1.
$$
\begin{moral}
    Note that in the continuous case $f(y|x) \neq \PP(Y=y|X=x)!$
\end{moral}

\begin{note}
    \vocab{Notation 1.} For any fixed $x \in \SX$ and for every set $A$
    $$
    \PP(Y \in A |X=x)) = \int_{A}f(y|x)dy
    $$
    The function $y \mapsto F(y|x):= \PP(Y \leq y|X=x)$ is called the \term{conditional cdf} of $Y$ given that $X=x$.    
    \\
    \vocab{Notation 2.} For every fixed $x \in \SX$ and for every real-valued function $g$
    $$
    \EE (g(Y)|X=x) = \int_{- \infty}^{\infty} g(y)f(y|x) dy.
    $$
\end{note}
\subsection{Independence}
From the definition of conditional pmf we can derive the joint pdf (or pmf) of $(X,Y)$ as 
$$
f_{X,Y}(x,y) = f_X(x)f_{Y|X}(y|x).
$$
\begin{definition}[Independent R.V's]
Let $(X,Y)$ be a random vector with joit pdf or pmf $f_{X,Y}(x,y)$, and marginal pdf's or pmf's $f_X(x)$ and $f_Y(y)$. If
$$
f_{X,Y}(x,y) = f_X(x)f_Y(y), \quad \text{for all } x,y \in \RR
$$
and we say that $X$ and $Y$ are \textit{independent}.
\end{definition}


\section{}
\subsection{Bivariate Transformations}
In this lecture our focus will be on computing the pdf of a bivariate random vector $(U,V)$ defined by
$$
U = g_1(X,Y), \quad V = g_2(X,Y)
$$
where $(X,Y)$ is a random vector with a known joint pdf $f(x,y)$ and $g_1,g_2$ are defined functions.

\begin{theorem}
Let $(X,Y)$ be a bivariate random vector with joint pdf $f_{X,Y}(x,y)$ and $\SA = \{
(x,y): f_{X,Y}(x,y)>0
\}$.
Let $g=(g_1,g_2): \SA \to \SB$ be a one-to-one transformation. Denote $g^{-1}:=h = (h_1,h_2)$ and let $J$ be the Jacobian of the transformation
    $$
    J = 
    \begin{bmatrix} \frac{\partial h_1}{\partial u} &
    \frac{\partial h_1}{\partial v} \\ 
    \frac{\partial h_2}{\partial u} & 
    \frac{\partial h_2}{\partial v}
    \end{bmatrix}
    $$
  Then the joint pdf of $(U,V)$ is 
  $$
  f_{U,V}(u,v) = f_{X,Y}(h_1(u,v),h_2(u,v)) \cdot |J|, \quad (u,v) \in \SB.
  $$
\end{theorem}
If the transformation $g$ is not one-to-one on $\SA$, but we can find a partition $\SA_1,\SA_2,...,\SA_k$ such thart $g = g^{(i)}:\SA_i \to \SB_i$ is one-to-one for each $i = 1,...,k$, then we can still write down a formula for the joint pdf of $(U,V) = g(X,Y)$:
$$
f_{U,V}(u,v) = \sum_{i=1}^{j}f_{X,Y}(h_1^{(i)}(u,v),h_2^{(i)}(u,v)) \cdot |J_i|\II_{(u,v) \in \SB_i}
$$
where $h^{(i)} = (h_1^{(i)},h_2^{(i)})$ are the inverse of $g^{(i)}$ and $J_i$ si the corresponding Jacobian.
\section{October 6, 2021}
\subsection{Hierarchical Models}
Thus far we have seen random variables which have single distributions, possibly depending on parameters. However, we can think of the parameter of a distribution as being a random variable, which itself has a distribution. 

\begin{example}[Binomial-Poisson hierarchy]
Classic example of  hierarchical model is the following: An insect lays a large number of eggs, each surviving with probability $p$. On average, how many eggs will survive? 
The ``large number" of eggs laid is a random variable, often taken to be Poisson$(\lambda)$. Furthermore, if we assume that each eggs survival is independent, we have Bernoulli trials. Therefore if $X =$ number of survivors and $Y=$ number of eggs laid, we have
$$
X|Y \sim \text{binomial}(Y,p)
$$
$$
Y \sim \text{Poisson}(\lambda),
$$
a hierarchical model.
\end{example}
\term{Solution:}\\
The random variable of interest, $X =$ number of survivors, has the distribution given by
\begin{align*}
    \PP(X=x) &= \sum_{y=0}^{\infty} \PP(X=x,Y=y) \\
             &= \sum_{y=0}^{\infty} \PP(X=x|Y=y) \PP(Y=y) \\
             &= \sum_{y=x}^{\infty} \left[\binom{y}{x}p^x(1-p)^{y-x} \right] \left[ \frac{e^{-\lambda \lambda^y}}{y!} \right] \\
             & \vdots \quad \text{(after some algebraic simplifications)} \\ 
             &= \frac{(\lambda p)^x e^{- \lambda}}{x!} e^{(1-p) \lambda}
\end{align*}
Thus, $X \sim$Poission($\lambda p$). Thus, any marginal inference on $X$ is with respect to a Poisson($\lambda p$) distribution, with $Y$ playing no part at all. Introduction $Y$ in the hierarchy was mainly to aid our understanding the model. 
\\
Now we can easily compute the expected value
$$
\EE[X]= \lambda p
$$
so, on average, $\lambda p$ eggs will survive. 
\\
\\
Sometimes calculations can be greatly simplified using the following theorem. Recall that $\EE(X|y)$ is a function of $y$ and $\EE(X|Y)$ is a random variable whose distribution depends on the value of $Y$.

\begin{theorem}

\end{theorem}
\section{October 13, 2021}
\subsection{Covariance and Correlation}

\begin{definition}[Covariance and Correlation]
    Let $X$ and $Y$ be random variables such that $\EE X^2 < \infty$, $\EE Y^2 < \infty$. The \vocab{covariance} of $X$ and $Y$ is defined by
    $$
    Cov(X,Y) = 
    \EE
    \left[(X - \EE X)(Y-\EE Y) \right]
    $$
    The \vocab{correlation} is defined as
    $$
    \rho_{X,Y} := Corr(X,Y) = 
    \frac{Cov(X,Y)}{\sqrt{\VV ar(X)\VV ar(Y)}}
    $$
    
\end{definition}
If $X$ and $Y$ are independent random variables, then Cov(X,Y) = 0. The converse is not true \textbf{in general}. However, the converse is true if $(X,Y)$ are \textbf{bivariate normal} distribution. 
\\
Furthermore, if $X$ and $Y$ are random variables and $a$ and $b$ are constants, then
$$
\VV ar(aX, bY) = a^2 \VV ar(X) + b^2 \VV ar(Y) + 2ab Cov(X,Y).
$$
\vocab{Consequence:} If $X$ and $Y$ are positively correlated(i.e, Cov($X,Y$) $\geq 0$), then 
$$
\VV ar(X +Y) \geq \VV ar(X) + \VV ar(Y)
$$
\term{Note:} A special case of positive dependence structure ``\textbf{association}". We say that $X$ and $Y$ are \textit{weakly associated} if
$$
Cov(g(X),h(Y)) \geq 0
$$
for any non-decreasing functions $g,h$ for which covariance exists.

\subsection{Inequalities}
The inequalities below, although often stated in terms of expectation, rely mainly on properties of numbers. They are all based on the following simple lemma.

\begin{lemma}
    Let $a$ and $b$ be any positive numbers, and let $p$ and $q$ any positive numbers (greater than 1) stratifying
    $$
    \frac{1}{p} + \frac{1}{q} = 1
    $$
    Then, 
    $$
    \frac{1}{p} a^p + \frac{1}{q} b^q \geq ab
    $$
    with equality of $a^p = b^q$.
\end{lemma}
\begin{proof}
    Fix, $b$ and consider the function 
    $$
    g(a) = \frac{1}{p} a^p + \frac{1}{q} b^q -ab
    $$
    To minimize $g(a)$, we differentiate and set equal to 0:
    $$
    \frac{d}{da}g(a) = 0 \implies a^{p-1} -b = 0 \implies a = a^{p-1} 
    $$
    We can also check the second derivative to establish that this is indeed a minimum. The value of the function at the minimum is 0. 
\end{proof}
\begin{theorem}[Holder's Inequality]
    Let $X$ and $Y$ be any two random variables, and let $p$ and $q$ satisfy. Then
    $$
    |\EE XY| \leq \EE|XY| \leq (\EE |X|^p)^{1/p}\EE |Y|^q)^{1/q}
    $$
\end{theorem}
\begin{proof}
    The first inequality follows from the fact that $-|XY| \leq XY \leq |XY|$ and theorem 2.2.5 in the textbook. To prove second inequality, define,
    $$
    a = \frac{|X|}{(\EE|X|^p)^{1/p}} \quad \text{and} \quad
    b = \frac{|Y|}{(\EE|Y|^q)^{1/q}}
    $$
    Applying the lemma above, and take the expectation of both side. The expectation of left-hand side is 1 and rearranging gives us the second inequality.
\end{proof}
\\
Perhaps the most famous special case of Holder's Inequality is the Cauchy-Schwartz ($p=2$).
\begin{theorem}[Cauchy-Schwartz Inequality]
    For any two random variables $X$ and $Y$,
    $$|\EE XY| \leq \EE |XY| \leq (\EE|X|^2)^{1/2} (\EE|Y|^2)^{1/2} 
    $$
\end{theorem}
\section{October 18, 2021}

\section{October 20, 2021}

\section{November 3, 2021}
\subsection{Sampling from Normal Distribution}
\begin{proposition}
    Let $X_1,...,X_n$ be a random sample from $N(\mu,\sigma^2)$ population, $\bar{X}$ be the sample mean and $S^2$ be the sample variance. Then, 
    \begin{enumerate}
        \item $\bar{X}$ and $S^2$ are \textbf{independent} random variables
        \item $\bar{X} \sim$ Normal($\mu,\sigma^2/n)$
        \item $(n-1)S^2/\sigma^2 \sim \chi_{n-1}^2$
    \end{enumerate}
\end{proposition}
Recall that if a random vector $(X,Y)$ has a normal distribution, then
$$
X,Y \quad \text{are independent } \iff \text{Cov}(X,Y) = 0
$$
We can actually generalize this idea to linear combinations of normal random variables. Let $X_1,...,X_n$ be random variables such that $X_i \sim N(\mu_i, \sigma_i^2)$ for each $i=1,..n$. Define the random vector $\boldsymbol{U} =(U_1,...,U_k)$ and $\boldsymbol{V} = (V_1,...,V_r)$ where
$$
U_i = \sum_{j=1}^n a_{ij}X_j, \quad i=1,...,k; \quad V_r = \sum_{j=1}^{n}b_{rj}X_j, \quad r=1,...,m
$$
and $ a_{ij}$, $b_{rj}$ are constants. Then,
$$
U_i \sim N \left( \sum_{j=1}^n a_{ij} \mu_i, \sum_{j=1}^n a_{ij}^2 \sigma_i^2 \right),
\quad
V_r \sim N \left( \sum_{j=1}^n b_{rj} \mu_i, \sum_{j=1}^n b_{rj}^2 \sigma_i^2 \right),
\quad
\text{Cov}(U_i,V_r) = \sum_{j=1}^n a_{ij} b_{rj} \sigma_j^2
$$
and, 
$$
U_i,V_r \text{ are independent } \iff \text{Cov}(U_i,V_r) = 0
$$
And this also implies that
$$
\boldsymbol{U},\boldsymbol{V} \text{ are independent random vectors } \iff U_i, V_r \text{ are independent } \forall i, \forall r.
$$
\vocab{Application:} If $X_1,...,X_n$ is a random sample from $N(\mu, \sigma^2)$ population, 
$$
X_j - \bar{X} \text{ and } \bar{X} \text{ are independent}
$$
for every $j=1,...,n$. From here we conclude that $S^2$ and $\bar{X}$ are independent. 

\subsection{The Derived Distributions: Student’s $t$ and Snedecor’s $F$}
\begin{definition}[Student's $t$ distribution]
We say that a random variable $T$ has \vocab{Student's t distribution} with $p$ degrees of freedom (and we write $T \sim t_p$) if its pdf is given by
$$
f(t) = \frac{\Gamma(p+1)/2}{\Gamma(p/2)} \cdot
\frac{1}{(1+t^2/p)^(p+1)/2}, \quad -\infty < t < \infty.
$$
\end{definition}
\textbf{Properties of the $t$ distribution}
\begin{enumerate}
    \item $t_1$ = Cauchy($0,1$)
    \item the graph of student's $t$ distribution is bell-shaped and symmetric around 0.
    \item We have
    $
    \EE T = 0, \quad \VV ar T = \frac{p}{p-2} \text{ if } p >2.
    $
\end{enumerate}
Let $U$ and $V$ be random variables such that $U \sim $ Normal(0,1) and $V \sim \chi_p^2$. Then
$$
T := \frac{U}{\sqrt{V/p}} \sim t_p.
$$
\vocab{Application.} Let $X_1,...,X_n$ be a random sample from Normal($\mu$, $\sigma^2$) population. Then
$$
T := \frac{\bar{X} - \mu}{S / \sqrt{n}} \sim t_{n-1}.
$$
\textbf{T is used to make inferences about $\mu$, when $\sigma^2$ is unknown.}
\begin{definition}[Snedecor’s F distribution]
We say that a continuous random variable $F$ has a Snedecor’s F distribution with $p$ and $q$ degrees of freedom (and we write $F \sim F_{p,q}$) if its pdf is given by
$$
f(x) = \frac{\Gamma((p+q)/2)}{\Gamma(p/2)\Gamma(q/2)} \cdot \left( 
\frac{p}{q}
\right)^{p/2} \cdot
\frac{x^{(p/2)-1}}{[1+(p/q)x]^{(p+q)/2)}}, \quad 0 < x < \infty.
$$
\end{definition}
Let $U$ and $V$ be independent random variables such that $U \sim \chi_{p}^2$ and $V \sim \chi_{q}^2$. Then
$$
F := \frac{U/p}{V/p} \sim F_{p,q}.
$$
\vocab{Application.} Let $\boldsymbol{X} = (X_1,...,X_n)$ be a random sample from Normal($\mu_X, \sigma_X^2$) and $\boldsymbol{Y} = (Y_1,...,Y_n)$ be a random sample from a Normal($\mu_Y,\sigma_{Y}^2$). Suppose that $\boldsymbol{X}$ and $\boldsymbol{Y}$ are independent. Then,
$$
F := \frac{S_{X}^2/ \sigma^2}{S_{Y}^2/ \sigma^2_Y} \sim F_{m-1,n-1}
$$
$F$ is used to make inferences about the ratio of $\sigma_{X}^2 / \sigma_{Y}^2 $

\section{November 8, 2021}
\subsection{Order Statistics}
We will consider a transformation that takes  $n$ RV's $X_1,...,X_n$ and essentially returns them in a sorted order. 
\begin{definition}
The \vocab{order statistics} of random variables $X_1,...,X_n$ are the random variables $X_{(1)},...X_{(n)}$, where
\begin{align*}
X_{1} & = min(X_{(1)},...X_{(n)})\\
X_{2} & = \text{ is the second-smallest of } X_1,...,X_n \\
& \vdots \\
X_{n-1} & = \text{ is the second-largest of } X_1,...,X_n \\
X_{n} & = max(X_{(1)},...X_{(n)}).
\end{align*}
\end{definition}
The sample range is a statistic defined as $R = X_{(n)} - X_{(1)}$. The midrange statistic is defined as $V = (X_{(1)} + X_{(n)})/2$. The \textit{sample median} is defined by
$$
M = 
\begin{cases}
    X_{\frac{n+1}{2}} \quad \text{if } n \text{ if odd} \\
   \frac{1}{2}(X_{\frac{n}{2}} + X_{\frac{n}{2} + 1} ) \quad \text{if } n \text{ if odd}
\end{cases}
$$
Its important to note that order statistics $X_{(1)},...X_{(n)}$ are random variables, and each $X_{(i)}$ is a function of the random sample $X_1,...,X_n$. Even if the original sample is independent, the order statistics are \textbf{dependent}! 
\begin{theorem}
Let $X_1,...,X_n$ be a random sample from discrete distribution with pmf $f(x)$. Suppose that the sample space of $X_1$ is a set $\{x_1,...,x_n \}$ such that $x_1 < x_2 < ...$ with CDF $F(x)$. Then the CDF of the $j^_{th}$ order statistic $X_{(j)}$ is. 
$$
\PP(X_{(j)} \leq x) = \sum_{k=j}^{n} {x \choose n} F(x) ^{k}(1 - F(x))^_{n-k}
$$
In addition to that, the pdf of the $j^_{th}$ order statistic $X_{(j)}$ is given by 
$$
f(X_{j}(x)) = \frac{n!}{(j-1)!(n-1)!}f(x)[F(x)]^{j-1}[1-F(x)]^{n-1}
$$
\end{theorem}

\section{November 10, 2021}
\subsection{Convergence Concepts}
The notion of letting sample size approach infinity can provide us with useful approximations, since it usually happens that expressions become simplified in the limit. 
\\
\subsection{Convergence in Probability}
\begin{definition}[Convergence in probability]
    A sequence of random variables, $X_1,X_2,...$ converges in probability to a random variable $X$ if, for every $\eps > 0$,
    $$
    \lim_{n \to \infty}\PP(|X_n -X| \geq \eps) = 0  \quad \text{or equivalently,} \quad  \lim_{n \to \infty}\PP(|X_n -X| < \eps) = 1
    $$
\end{definition}
The law of large number asserts that as $n$ grows, the sample mean $\bar{X}_n$ converges to true mean $\mu$. Law of larger number has two versions (weak and strong), the difference in the two lies in what is mean for a sequence of random variables to converge to a number. 

\begin{theorem}[Weak Law of Large numbers]
    Let $X_1,X_2,...$ be iid random variables with $\EE X_i = \mu$ and $\VV ar X_i = \sigma^2 < \infty$. Define, $\bar{X}_n = \frac{1}{n}\sum_{i=1}^{n}X_i$. Then, for every $\eps > 0$,
    $$
    \lim_{n \to \infty} \PP(|\bar{X}_n - \mu| < \eps ) = 1,
    $$
    that is, $\bar{X}_n$ converges in probability to $\mu$.
\end{theorem}
We can extend the definition of convergence in probability to functions of random variables. Suppose that $X_1,X_2,...$ converges in probability to a random variable $X$ or to a constant $a$ and $h$ is a continuous function. Then $h(X_1), h(X_2),...$ converges in probability to $h(X)$.
\\
We can use the above fact to easily prove that since $S_n^2 \to \sigma^2 \implies S_n = \sqrt{S_n} \to \sigma$.
\begin{note}
    \vocab{Properties of convergence in probability:}
    \begin{enumerate}
        \item if $X_n \xrightarrow{p} X$ and $X_n \xrightarrow{p} Y$, then $X =Y$ asymptotically. 
        \item if $X_n \xrightarrow{p} X$ and if $Y_n \xrightarrow{p} Y$, then if $X_n +Y_n \xrightarrow{p} X + Y$.
        \item If $X_n \xrightarrow{p} X$, $X_n \xrightarrow{p} X$, and $g(x,y)$ is a continuous function, then $g(X_n,Y_n) \xrightarrow{p} g(X,Y)$. 
    \end{enumerate}
\end{note}
\subsection{Almost Sure Convergence and Strong Law of Large Numbers}
Almost sure convergence is stronger than convergence in probability. It is similar to point wise convergence of a sequence of functions.
\begin{definition}[Almost surely convergence]
    A sequence of random variables, $X_1,X_2,...$ converges almost surely to a random variable $X$ if, for every $\eps > 0$,
    $$
    \PP(\lim_{n \to \infty}|X_n - X|< \eps ) = 1.
    $$
\end{definition}
Lets try to understand this definition a bit deeper. A random-variable is just a real-valued function defined on a sample space $S$. Let $s \in S$, then $X_{n}(s)$ and $X(s)$ are all functions defined on $S$. The definition is saying that $X_n$ converges to $X$ almost surely if the functions $X_{n}(s)$ converges to $X(s)$.\\
\\
\term{Note:} \textit{Almost surely convergence} implies convergence in probability, however converse is not true. We say that the sequence of $\{ \hat{\theta}_n \}_n$ is a \textit{strongly consistent} estimators for the parameter $\theta$ if $\hat{\theta}_n \xrightarrow{a.s} \theta$.
\begin{theorem}[The Strong Law of Large Numbers]
    Let $\{ \bar{X}_n \}_n$ be a sequence of iid random variables. Suppose that $\EE|X_1| \leq \infty$ and $\EE X_1 = \mu$. Then 
    $$
    \bar{X}_n \xrightarrow{a.s} \mu 
    $$
    In other words, sample mean $\bar{X}_n$ converges to the true mean $\mu$ pointwise as $n \to \infty$.
\end{theorem}
The law of large number plays a crucial role in simulations and statistics. Lets say we generate data from a large number of i.i.d samples of an experiment, either using a computer simulation or relation world. If we employ proportion of times an event occurred to approximate the probability of the event, we are \textit{implicitly} applying the Law of Large Numbers.
\subsection{Convergence in Distribution and Central Limit Theorem}
\begin{definition} [Convergence in Distribution]
    We say that $\{ X_n \}_{n \geq 1}  = \{ X_1,X_2,... \}$ of random variables \term{converges in distribution} to a random variable $X$ if 
    $$
    F_{X_n}(x) = \PP(X_n \leq x) \xrightarrow{n \to \infty} \PP(X \leq x) = F_{X}(x) \quad \text{for all } x \in \RR \
    $$
    such that $F_X$ is continuous.\\
    We say that sequence $\hat{\theta}_{n \geq 1}$ of estimators of $\theta$ is \vocab{asymptotically normal} for $\theta$ if 
    $$
    \sqrt{n}(\theta_n - \theta) \xrightarrow{d} Z \sim N(0,\sigma^2) \quad \text{ for some } \sigma^2 >0.
    $$
\end{definition}
Lets try to connect the ideas from previous sections to above definition. The law of large numbers essentially says that if we have $X_1,X_2,X_2,...$ i.i.d with mean $\mu$ and variance $\sigma^2$, $\bar{X}_n \to \mu$ as $n \to \infty$ with probability $1$. But what is the distribution of $\bar{X}_n$ along the way to becoming a constant? This is where the Central Limit Theorem (CLT) comes into play. 
\begin{theorem}[Central Limit Theorem]
    Let $\{ X_n \}_{n \geq 1}  = \{ X_1,X_2,... \}$ of random variables, assume that $\EE X_i = \mu$ is finite and $\VV ar(X_i) = \sigma^2 < \infty$.
    Then 
    $$
    \sqrt{n}(\bar{X}_n - \mu) \xrightarrow{d} Z \sim N(0,\sigma^2)
    $$
\end{theorem}

If $X_n \xrightarrow{d} X$ and $X_n \xrightarrow{d} a$, then
    $$
    X_n + Y_n \xrightarrow{d} X + a \quad \text{and} \quad
    X_nY_n \xrightarrow{d} Xa
    $$

We say that $\{ X_n \}_{n \geq 1}$ is a sequence of asymptotically normal estimators of $\mu$.
\begin{note}
    \vocab{Delta Method.} Let $\{ \hat{\theta}_n \}_n$ be a sequence of asymptotically normal estimators of $\theta$. Recall that asymptotically normal means that random variables satisfies $\sqrt{n}(Y_n - \theta) \to n(9,\sigma)$ in distribution. For a given function $g$ and specific value of $\theta$, suppose that $g'(\theta)$ exists and is not $0$. Then 
    $$
    \sqrt{n}[g(Y_n)-g(\theta)] \to n(0,\sigma^2[g'(\theta)]^2) \quad \text{in distribution.} 
    $$ 
    
\end{note}
\section{November 15, 2021}
\subsection{Inference}
Our objective is now to use the information in the sample $X_1,...,X_n$ to make inferences about the unknown parameter.
\\
\\ 
For $j = 1,..m$, let $T_j$ be measurable functions defined on $\RR^n \to \RR$ and not depending on $\theta$, and let $\boldsymbol{T} = (T_1,...,T_m)'$. Then
    $$
    \boldsymbol{T}(X_1,...,X_n) = \left(
    T_1(X_1,...,X_n),...,T_m(X_1,...X_n)
    \right)'
    $$
is called a \vocab{m-dimensional statistic}.
Any statistic T($\boldsymbol{X}$) defines a form of data reduction by partitioning the sample space $\SX$. If we only use the value of the statistic, T($\boldsymbol{x}$), rather than the entire sample $\boldsymbol{x}$, the two samples $\boldsymbol{x}$ and $\boldsymbol{y}$ will be treated equally if T($\boldsymbol{x}) =$ T($\boldsymbol{y}$) 
\subsection{Sufficiency Principle}
\begin{definition}
    Let $\boldsymbol{X} = (X_1,...,X_n)$ be a random sample with a join pdf or pmf denoted by $f(\boldsymbol{x}|\theta)$. Let T = T($\boldsymbol{x}$)
    be a statistic based on this sample with pdf (or pmf) denoted by $f_T(t|\theta)$. We say that T is a \vocab{sufficient statistic} for $\theta$ if for any $t$ such that $f_T(t|\theta) > 0$, the conditional pdf (or pmf) of $\boldsymbol{X}$ given $T=t$ does not depend on $\theta$.
    \end{definition}
\term{Remark.} The joint pdf (or pmf) of $(\boldsymbol{X},T)$ is 
    $$
    f_{X,T}(\boldsymbol{x},t| \theta) =
        \begin{cases}
            f(\boldsymbol{x}| \theta) \quad \text{if } t= T(\boldsymbol{x}) \\
            0 \quad \text{otherwise}
        \end{cases}
    $$
    
\begin{theorem}[Factorization Theorem]
    Let $\boldsymbol{X} = (X_1,...,X_n)$ be a random sample with a join pdf or pmf denoted by $f(\boldsymbol{x}|\theta)$. Let
    $$\boldsymbol{T} = \boldsymbol{T}(X_1,...,X_n) = \left(
    T_1(X_1,...,X_n),...,T_k(X_1,...X_n)
    \right)'$$ be a $k-$dimensional statistic. Then $\boldsymbol{T}$ is a sufficient statistic for $\boldsymbol{\theta}$ if and only if 
    $$
    f(\boldsymbol{x}|\theta) = g(\boldsymbol{T}|\boldsymbol{\theta})\cdot h(\boldsymbol{x}) \quad \text{ for all } \boldsymbol{x} \in \SX.
    $$

\end{theorem}
\section{November 17, 2021}
\subsection{Factorization Theorem Continued}
\begin{example}
    Let $X_1,...,X_n$ be independent random variables such that $$X_k \sim Unif(k(\theta -1), k(\theta+1)).$$
    
    Show that $ \left ( \displaystyle \underset{1 \leq k \leq n}{\mathrm{min}} \frac{x_k}{k}, \underset{1 \leq k \leq n}{\mathrm{max}} \frac{x_k}{k}
    \right )$ is a $2-$dimensional sufficient statistic. 
    
\end{example}

\begin{theorem}
    Let $\boldsymbol{X} = (X_1,...,X_n)$ be a random sample from a pdf(or pmf) which belongs to the following exponential family
    $$
    f_X(x|\theta) = h(x)c(\theta)exp 
    \left \{ 
    \sum_{i=1}^{k}w_i(\theta)t_i(x)
    \right \}
    $$
    with $\theta = (\theta_1,...,\theta_d)$. Suppose that $d \leq k$. Define 
    $$T_i = T_i(\boldsymbol{X}) = \sum_{j=1}^{n}T_i(X_j)
    \quad \forall i=1,...,k.$$
    Then $\boldsymbol{T} = (T_1,...,T_k)$ is a sufficient statistic for $\theta$.
\end{theorem}
\begin{proof}
    The joint pdf of $\boldsymbol{X}$ is:
    \begin{align*}
    f(\boldsymbol{x}|\theta) & = 
    \prod_{j=1}^{n}f_{x_j}(x_j \theta) 
    = \prod_{j=1}^{n} \left[
    h(x_j)c(\theta) \left \{
    \sum_{i=1}^{k}w_i(\theta)t_i(x_j)
    \right\}
    \right ] \\
    & = \prod_{j=1}^{n}h(x_j)[c(\theta)]^n \mathrm{exp} \left \{
    \sum_{i=1}^{k}w_i(\theta)t_i(x_j)
    \right\}
    \end{align*}
    Clearly, 
    $$
    \underbrace{ \prod_{j=1}^{n}h(x_j)}_{h'(x)}
    \underbrace{[c(\theta)]^n\mathrm{exp} \left \{
    \sum_{i=1}^{k}w_i(\theta)t_i(x_j)
    \right\}}_{g(T_1(\boldsymbol{x}),...,T_k(\boldsymbol{x})|\theta)}
    $$
    Thus, by the Factorization theorem $(T_1(\boldsymbol{x}),...,T_k(\boldsymbol{x})|\theta)$ is a sufficient statistic for $\theta$.
\end{proof}
\term{Remark:} If $\boldsymbol{T}$ is a sufficient statistic for $\theta$, then any one-to-one transformation of $\boldsymbol{S} = \pi(\boldsymbol{T})$ is also a sufficient statistic for for $\theta$, $\forall \pi$ one-to-one maps. \textbf{Therefore, a sufficient statistic is not unique}. 
\begin{definition}
    A sufficient statistic is $\boldsymbol{T} = (T_1,...,T_k)$ is called a \vocab{minimal sufficient statistic} for $\theta$ if, for any other sufficient statistic $\boldsymbol{T}^* = (T_1^*,...,T_k^*)$, there exists a function $\phi$ such that $\boldsymbol{T} = \phi(\boldsymbol{T}^*)$. This is equivalent to saying that
    $$
    T^*(\boldsymbol{x}) = T^*(\boldsymbol{y}) \Longrightarrow T(\boldsymbol{x}) = T(\boldsymbol{y})
    $$
\end{definition}
\term{Note:} A minimal sufficient statistic may \textbf{NOT} be unique.
\begin{theorem}[Lehman and Scheffe]
    Let $\boldsymbol{X} = (X_1,...,X_n)$ be a random sample from a pdf(or pmf) which belongs to the following exponential family $f_X(x|\theta)$. Let $T = T(\boldsymbol{X})$ be a statistic which satisfies the following condition
    $$
    \frac{f(\boldsymbol{x}|\theta)}{f(\boldsymbol{y}|\theta)}\quad \textbf{does not depend on } \theta
    \iff T(\boldsymbol{x}) =  T(\boldsymbol{y})
    $$
    Then $T$ is a minimal sufficient statistic. 
\end{theorem}
\begin{proof}
    Proof is given in the text book.
\end{proof}
\section{November 22, 2021}
\subsection{The Sufficiency Principle (Continued)}
So far, we have covered sufficient statistics which in a sense contain all the information about $\theta$ that is available in the sample. Next we look at a statistic which is quite the opposite.
\begin{definition}[Ancillary Statistic]
    A statistic $S(\boldsymbol{X})$ whose distribution does not depend on the parameter
    $\theta$ is called an \vocab{ancillary} statistic.
\end{definition}
An ancillary statistic contains no information about $\theta$! An ancillary statistic has a fixed and known distribution that is unrelated to $\theta.$
\begin{example}[Location and Scale family ancillary statistic]
    Let $X_1,...,X_n$ be iid observations from a location parameter family with pdf $g(x|\mu) = f(x-\mu)$ where $f$ is a standard pdf. Then we will show that
    $$
    \begin{cases}
        R = X_{(n)} - N_{(1)} \quad \text{is ancillary statistic for } \mu. \\
        S^2 = \frac{1}{n-1} \sum_{i=1}^{n}(X_i - \Bar{X})^2 \quad \text{is ancillary statistic for } \mu.
    \end{cases}
    $$ 
    In addition to that if we have random sample $\boldsymbol{X} = (X_1,...,X_n)$ from a \textit{scale family} with pdf $g(x|\theta) = \frac{1}{\sigma}f(x/\sigma)$, where $f(z)$ is the standard pdf of the family. Then
    $$
    T(\boldsymbol{X}) = \left( 
    \frac{X_1}{X_n},...,\frac{X_{n-1}}{X_n} 
    \right)
    \text{ is an ancillary statistic for }\sigma
    $$
    In particular $\bar{X}/X_n$ is an ancillary statistic for $\sigma$.
    \begin{proof}
        
    \end{proof}
\end{example}
\begin{definition}[Complete statistic]
    Let $f(t|\theta)$ be a family of pdf or omfs for a statistic $T(\boldsymbol{X})$. The family of probability distributions is called \vocab{complete} if $\EE_{\theta}(T) = 0$ for all $\theta$ implies that $\PP_{\theta}(g(T) =0) =1$ for all $\theta$. Equivalently, $T(\boldsymbol{X})$ is called a \vocab{complete statistic}.
    \\
    In other words, let $T$ be the statistic whose range is $\ST$, it is called complete if
    $$
    \EE g(T) = 0 \text{ for all } \theta \implies g(t) =0 \text{ for all } t \in \ST
    $$
\end{definition}
\begin{theorem}[Basu's Theorem]
    If $T(\boldsymbol{X})$ is a complete ad minimal sufficient statistic, the $T(\boldsymbol{X})$ is independent of every ancillary statistic. 
\end{theorem}
\begin{proof}
    The proof for the discrete case is given on page 287 of the textbook, review it!
\end{proof}
\begin{theorem}[Complete Statistics in the exponential families]
    Let $\boldsymbol{X} = (X_1,...,X_n)$ be iid observations from an exponential family with pdf or pmf of the form
    $$
        f(x|\theta) = h(x)c(\theta)\text{exp} \left (
            \sum_{j=1}^{k}w(\theta_j)t_j(x),
        \right) 
    $$
    where $\boldsymbol{\theta} = (\theta_1,...,\theta_k)$. Then the statistic 
    $$
    T(\boldsymbol{X}) = \left (
        \sum_{i=1}^{n}t_1(X_i), \sum_{i=1}^{n}t_2(X_i),...,\sum_{i=1}^{n}t_k(X_i),
    \right )
    $$
    is complete as long as the parameter space $\Theta$ contains an open set in $\RR^k$.
\end{theorem}
\begin{proof}
    Proof is omitted from this course.
\end{proof}
\subsection{Likelihood Principle}
\begin{definition}[Likelihood function]
    Let $f(\boldsymbol{x}|\theta)$ denote the joint pdf or pmf of the sample $\boldsymbol{X} = (X_1,...,X_n)$. Then, given that $\boldsymbol{X} = \boldsymbol{x}$ is observed, the function of $\theta$ defined by $$
    L(\theta|\boldsymbol{x}) = f(\boldsymbol{x}|\theta)
    $$
    is called the \vocab{likelihood function}.
\end{definition}
\textbf{Likelihood Principle:} If $\boldsymbol{x}$ and $\boldsymbol{y}$ are two sample points such that $L(\theta|\boldsymbol{x}) \propto L(\theta|\boldsymbol{y})$, i.e, $\exists C(\boldsymbol{x},\boldsymbol{y})$ constant such that
$$
    L(\theta|\boldsymbol{x}) = C(\boldsymbol{x},\boldsymbol{y}) L(\theta|\boldsymbol{y}) \quad \text{for all } \theta,
$$
then the conclusions drawn from $\boldsymbol{x}$ and $\boldsymbol{y}$ should be identical. 
We define an \textbf{experiment} $E$ to be a triple $(\boldsymbol{X}, \theta, \{f(\boldsymbol{x}|\theta)\})$, wher $\boldsymbol{X}$ is a random vector with pmf $f(\boldsymbol{x}|\theta)$ for some $\theta \in \Theta$. An experimenter knowing what experiment $E$ was performed hand having observed a particular $\boldsymbol{X} = \boldsymbol{x}$, will make some inference or draw soem conclusion about $\theta$. We denote this conclusion by Ev$(E,\boldsymbol{x})$, which stands for \textit{evidence about $\theta$ arising from $E$ and $\boldsymbol{x}$}.
\\
\textbf{Formal sufficiency Principle:} Consider experiment $E = (\boldsymbol{X}, \theta,\{f(\boldsymbol{x}|\theta)\})$ and suppose that is $T(\boldsymbol{X})$ a sufficient statistic for $\theta$. If $\boldsymbol{x}$ and $\boldsymbol{y}$ are two samples such that $T(\boldsymbol{x}) = T(\boldsymbol{y})$ then
$$
\text{Ev}(E,\boldsymbol{x}) = \text{Ev}(E,\boldsymbol{y})
$$
\textbf{Formal Likelihood Principle:} Suppose that we have two experiments, $E_1 = (\boldsymbol{X}_1, \theta, \{f(\boldsymbol{x}_1|\theta)\})$ and $E_2 = (\boldsymbol{X}_2, \theta, \{f(\boldsymbol{x}_2|\theta)\})$, where we have the unknown parameter $\theta$ is the same for both experiments. If $\boldsymbol{x}$ is a sample from $E_1$ and $\boldsymbol{y}$ is a sample from $E_2$ such that 
$$
    L(\theta|\boldsymbol{x}) = C(\boldsymbol{x},\boldsymbol{y}) L(\theta|\boldsymbol{y}) \quad \text{for all } \theta,
$$
then,
$$
\text{Ev}(E_1,\boldsymbol{x}) = \text{Ev}(E_2,\boldsymbol{y})
$$
\section{November 24, 2021}
\subsection{Likelihood Principle}

\section{November 29, 2021}
\subsection{Invariance property of MLE}
Suppose that the distribution is index by the parameter $\theta$, if are interested in estimating some function of $\theta$, say $\tau(\theta)$.
If we let $\eta = \tau(\theta)$, then the inverse function $\theta = \tau^{-1}(\eta)$ is well-defined, and we can express the likelihood function as a function of $\eta$
$$
L^{*}(\eta|\boldsymbol{x}) \prod_{i=1}^{n}f(x_i|\tau^{-1}(\eta)) = L(\tau^{-1}(\eta)|\boldsymbol{x})
$$
and,
$$
\underset{\eta}{\text{sup}}L^{*}(\eta|\boldsymbol{x}) = \underset{\eta}{\text{sup}}L(\tau^{-1}(\eta)|\boldsymbol{x})
= \underset{\eta}{\text{sup}}L(\theta|\boldsymbol{x})
$$ 
The \vocab{induced likelihood} of $\eta$ is defined by 
$$
\underset{\eta}{\text{sup}}L^{*}(\eta|\boldsymbol{x}) =
\underset{\theta: \tau(\theta) = \eta}{\text{sup}}L(\theta|\boldsymbol{x})
$$
The value of $\Hat{\eta}$ which maximized $L^{*}(\eta|\boldsymbol{x})$ is called the MLE of $\eta$:
$$
\underset{\eta}{\text{sup}}L^{*}(\eta|\boldsymbol{x}) =L^{*}(\Hat{\eta}|\boldsymbol{x})
$$
\section{December 1, 2021}
\subsection{Methods of evaluating estimators}
The general topic of evaluating statistical procedures is part of a branch of statistics called decision theory. In this section we look at some basic criteria for evaluating estimators.
    
\begin{definition}[Mean square error (MSE)]
    The \vocab{mean square error} of an estimator $W$ of a parameter $\theta$ is the function of $\theta$ defined by $\EE_{\theta}(W-\theta)^2$.
\end{definition}

\begin{definition}[Bias]
    The bias of an estimator $T$ for a parameter $\theta$ is defined by 
    $$
    \text{Bias}_{\theta}(T) = \EE_{\theta}T -\theta
    $$
\end{definition}
\textbf{Note:} $MSE$ incorporates two components, one measuring the variability and the other measuring bias.
$$
\text{MSE}_{\theta} = \VV ar_{\theta}(T) + (\text{Bias}_{\theta}(T))^2
$$
\begin{example}
    Let $X-1,...,X_n$ be iid $n(\mu,\sigma^2)$. The statistics $\bar{X}$ and $\S^2$ are both biased estimators since
    $$
    \EE \bar{X} = \mu \quad, \EE S^2 = \sigma^2, \quad \text{for all $\mu$ and $\sigma$}.
    $$
    The MSEs of the estimators are given by
    $$
    \EE(\bar{X} - \mu)^2 = \VV ar \bar{X} = \frac{\sigma}{n}.
    $$
    $$
    \EE(S^2 \sigma^2)^2 = \VV ar S^2
    $$
\end{example}
\subsection{Best unbiased estimators (UMVUE)}
If we compare estimators based on MSE, there is no single \textit{"best MSE"} estimator. The reason is that the class of all estimators is too large of a class. Therefore by placing certian restrictions on our estimators, we can limit the class of estimators.  
\begin{definition}[Best unbiased estimator]
    An estimator $W^*$ is the \vocab{best unbiased estimator} of $\tau(\theta)$ if it satisfies $\EE_{\theta}W^* = \tau(\theta)$ for all $\theta$ and, for any for any other estimator $W$ with $\EE_{\theta}W = \tau(\theta)$, we have $\VV ar_{\theta}W^* \leq \VV ar_{\theta} W$ for all $\theta$. $W^*$ is also called a \vocab{uniform minimum variance unbiased estimator (UMVUE)}
\end{definition}
\section{December 6, 2021}
\subsection{methods of evaluating estimators (continued)}
The general topic of evaluating statistical procedures is part of a branch of statistics called decision theory. In this section we look at some basic criteria for evaluating estimators.
    
\begin{definition}[Mean square error (MSE)]
    The \vocab{mean square error} of an estimator $W$ of a parameter $\theta$ is the function of $\theta$ defined by $\EE_{\theta}(W-\theta)^2$.
\end{definition}

\subsection{Sufficiency and Unbiasedness}
\input{lectures/lecture23}








\end{document}
