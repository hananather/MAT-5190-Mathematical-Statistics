\section{November 29, 2021}
\subsection{Invariance property of MLE}
Suppose that the distribution is index by the parameter $\theta$, if are interested in estimating some function of $\theta$, say $\tau(\theta)$.
If we let $\eta = \tau(\theta)$, then the inverse function $\theta = \tau^{-1}(\eta)$ is well-defined, and we can express the likelihood function as a function of $\eta$
$$
L^{*}(\eta|\boldsymbol{x}) \prod_{i=1}^{n}f(x_i|\tau^{-1}(\eta)) = L(\tau^{-1}(\eta)|\boldsymbol{x})
$$
and,
$$
\underset{\eta}{\text{sup}}L^{*}(\eta|\boldsymbol{x}) = \underset{\eta}{\text{sup}}L(\tau^{-1}(\eta)|\boldsymbol{x})
= \underset{\eta}{\text{sup}}L(\theta|\boldsymbol{x})
$$ 
The \vocab{induced likelihood} of $\eta$ is defined by 
$$
\underset{\eta}{\text{sup}}L^{*}(\eta|\boldsymbol{x}) =
\underset{\theta: \tau(\theta) = \eta}{\text{sup}}L(\theta|\boldsymbol{x})
$$
The value of $\Hat{\eta}$ which maximized $L^{*}(\eta|\boldsymbol{x})$ is called the MLE of $\eta$:
$$
\underset{\eta}{\text{sup}}L^{*}(\eta|\boldsymbol{x}) =L^{*}(\Hat{\eta}|\boldsymbol{x})
$$