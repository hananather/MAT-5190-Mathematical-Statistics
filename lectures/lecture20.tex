\section{November 29, 2021}
\subsection{Invariance property of MLE}
Suppose that the distribution is index by the parameter $\theta$, if are interested in estimating some function of $\theta$, say $\tau(\theta)$.
If we let $\eta = \tau(\theta)$, then the inverse function $\theta = \tau^{-1}(\eta)$ is well-defined, and we can express the likelihood function as a function of $\eta$
$$
L^{*}(\eta|\boldsymbol{x}) \prod_{i=1}^{n}f(x_i|\tau^{-1}(\eta)) = L(\tau^{-1}(\eta)|\boldsymbol{x})
$$
and,
$$
\underset{\eta}{\text{sup }}L^{*}(\eta|\boldsymbol{x}) = \underset{\eta}{\text{sup }}L(\tau^{-1}(\eta)|\boldsymbol{x})
= \underset{\eta}{\text{sup }}L(\theta|\boldsymbol{x})
$$ 
The \vocab{induced likelihood} of $\eta$ is defined by 
$$
\underset{\eta}{\text{sup }}L^{*}(\eta|\boldsymbol{x}) =
\underset{\theta: \tau(\theta) = \eta}{\text{sup}}L(\theta|\boldsymbol{x})
$$
The value of $\Hat{\eta}$ which maximized $L^{*}(\eta|\boldsymbol{x})$ is called the MLE of $\eta$:
$$
\underset{\eta}{\text{sup }}L^{*}(\eta|\boldsymbol{x}) =L^{*}(\Hat{\eta}|\boldsymbol{x})
$$
\begin{theorem}[Invariance property of MLEs]
    If $\hat{\theta}$ is the MLE of $\theta$, then for any function $\tau(\theta)$, the MLE of $\tau(\theta)$ is $\tau(\hat{\theta})$.
\end{theorem}
\begin{proof}
    Let $\hat{\eta}$ denote the value that maximizes $L^{*}(\eta|\boldsymbol{x})$. Our objective is to show that $L^{*}(\eta|\boldsymbol{x}) = L^{*}(\tau(\hat{\theta})|\boldsymbol{x})$. Since,
    \begin{align*}
        L^{*}(\hat{\eta}|\boldsymbol{x}) &= \underset{\eta}{\text{sup}}
        \underset{\Large \theta: \tau(\theta) = \eta}{\text{sup}} L(\theta|\boldsymbol{x})\\
        &= \underset{\theta}{\text{sup }}L(\theta|\boldsymbol{x}) \\ 
        & = L(\hat{\theta}|\boldsymbol{x})
    \end{align*}
\end{proof}
We can see that using this theorem that MLE of $\theta^2$ is $\bar{X}^2$, and for a more complicated function such as $\sqrt{p(1-p)}$, where $p$ is the binomial probability, the MLE is $\sqrt{\hat{p}(1-\hat{p} )}$.
\subsection{Bayes Estimators}
Denote the prior distribution $\pi(\theta)$ and the sampling distribution by $f(\boldsymbol{x}|\theta)$, then the posterior distribution is given by 
$$
\pi(\theta|\boldsymbol{x}) = \frac{f(\boldsymbol{x}|\theta)\pi(\theta)}{m(\boldsymbol{x})} \quad 
\quad \text{(Bayes' Rule!)}
$$
where $m(\boldsymbol{x})$ is the marginal distribution of $\boldsymbol{X}$, that is 
$$
    m(\boldsymbol{x}) = \int f(\boldsymbol{x}|\theta)\pi(\theta)d\theta
$$

\begin{definition}
    Let $\SF$ be a class of distributions for $X$. A class $\Pi$ of prior distributions is a \vocab{conjugate family} for $\SF$ if the posterior distribution is in the class $\Pi$ for all $f \in \SF$, i.e,
    $$
    \pi(\theta|\boldsymbol{x})\quad \text{ for all } \pi \in \Pi, f \in \SF
    $$
\end{definition}