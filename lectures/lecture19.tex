\section{November 24, 2021}
\subsection{Likelihood Principle}
\begin{definition}[Likelihood function]
    Let $f(\boldsymbol{x}|\theta)$ denote the joint pdf or pmf of the sample $\boldsymbol{X} = (X_1,...,X_n)$. Then, given that $\boldsymbol{X} = \boldsymbol{x}$ is observed, the function of $\theta$ defined by $$
    L(\theta|\boldsymbol{x}) = f(\boldsymbol{x}|\theta)
    $$
    is called the \vocab{likelihood function}.
\end{definition}
\textbf{Likelihood Principle:} If $\boldsymbol{x}$ and $\boldsymbol{y}$ are two sample points such that $L(\theta|\boldsymbol{x}) \propto L(\theta|\boldsymbol{y})$, i.e, $\exists C(\boldsymbol{x},\boldsymbol{y})$ constant such that
$$
    L(\theta|\boldsymbol{x}) = C(\boldsymbol{x},\boldsymbol{y}) L(\theta|\boldsymbol{y}) \quad \text{for all } \theta,
$$
then the conclusions drawn from $\boldsymbol{x}$ and $\boldsymbol{y}$ should be identical. 
We define an \textbf{experiment} $E$ to be a triple $(\boldsymbol{X}, \theta, \{f(\boldsymbol{x}|\theta)\})$, wher $\boldsymbol{X}$ is a random vector with pmf $f(\boldsymbol{x}|\theta)$ for some $\theta \in \Theta$. An experimenter knowing what experiment $E$ was performed hand having observed a particular 
