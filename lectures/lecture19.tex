\section{November 24, 2021}
\subsection{Equivariance Principle}
\textbf{Equivariance Principle:} If $\boldsymbol{Y} = g(\boldsymbol{X})$ is a change of measurement scale such that the model for $\boldsymbol{Y}$ has the same formal structure as the model for $\boldsymbol{X}$,  then inference procedure should be both measurement equivariant and formally equivariant. 
\begin{definition}[Group transformation]
    A set $\SG$ of functions, $\{ g(\boldsymbol{x}: g \in \SG) \}$, of the form $g: \SX \to \SX$ is called a \vocab{group transformation} of $\SX$ if
    \begin{enumerate}[(i)]
        \item (Inverse) $\forall g \in \SG$, $\exists g^{-1} \in \SG$ such that $g\circ g^{-1} = e$, where $e: \SX \to \SX$ is the identity $e(x) = x$. 
        \item (Composition) For every $g,g^{'} \in \SG$, $g \circ g^{'} \in \SG$
        \item (Identity) The identity, $e(\boldsymbol{x})$ is an element of $\SG$.
    \end{enumerate}
\end{definition}
\begin{definition}[Invariant]
    Let $\SF = \{f(\boldsymbol{x}|\theta): \theta \in \Theta \}$ be a set of pdf or pmfs for $\boldsymbol{X}$, and let $\SG$ be a group of transformations of the  sample space $\SX$. Then $\SF$ is \vocab{invariant under the group $G$} if for every $\theta \in \Theta$ and $g \in \SG$ there exists a unique $\theta^{'} \in \Theta$ such that $\boldsymbol{Y} = g(\boldsymbol{X})$ has the distribution $f(\boldsymbol{y}|\theta^{'})$ if $\boldsymbol{X}$ has the distribution $f(\boldsymbol{x}|\theta)$.
\end{definition}
The equivariance principle essentially says that inference based on $\boldsymbol{X}$ should be the same as inference based on $\boldsymbol{Y}$.
\begin{example}[Location family is invariant]

\end{example}
\begin{example}[Scale family is invariant]

\end{example}
\subsection{Methods of Moments}
A statistic is a function of the random sample, and we generally use such functions to approximate the value of the unknown parameter $\theta$, which is underlying the parameter for the underlying distribution of the sample) called \vocab{estimator}. Once we have observed a particular realization of $\boldsymbol{X}$, (the sample $\boldsymbol{x}$), we refer to the value $T(\boldsymbol{x})$ of the estimator for that particular realization as \vocab{estimate for $\theta$}.