\section{November 8, 2021}
\subsection{Order Statistics}
We will consider a transformation that takes  $n$ RV's $X_1,...,X_n$ and essentially returns them in a sorted order. 
\begin{definition}
The \vocab{order statistics} of random variables $X_1,...,X_n$ are the random variables $X_{(1)},...X_{(n)}$, where
\begin{align*}
X_{1} & = min(X_{(1)},...X_{(n)})\\
X_{2} & = \text{ is the second-smallest of } X_1,...,X_n \\
& \vdots \\
X_{n-1} & = \text{ is the second-largest of } X_1,...,X_n \\
X_{n} & = max(X_{(1)},...X_{(n)}).
\end{align*}
\end{definition}
The sample range is a statistic defined as $R = X_{(n)} - X_{(1)}$. The midrange statistic is defined as $V = (X_{(1)} + X_{(n)})/2$. The \textit{sample median} is defined by
$$
M = 
\begin{cases}
    X_{\frac{n+1}{2}} \quad \text{if } n \text{ if odd} \\
   \frac{1}{2}(X_{\frac{n}{2}} + X_{\frac{n}{2} + 1} ) \quad \text{if } n \text{ if odd}
\end{cases}
$$
Its important to note that order statistics $X_{(1)},...X_{(n)}$ are random variables, and each $X_{(i)}$ is a function of the random sample $X_1,...,X_n$. Even if the original sample is independent, the order statistics are \textbf{dependent}! 
\begin{theorem}
Let $X_1,...,X_n$ be a random sample from discrete distribution with pmf $f(x)$. Suppose that the sample space of $X_1$ is a set $\{x_1,...,x_n \}$ such that $x_1 < x_2 < ...$ with CDF $F(x)$. Then the CDF of the $j^{th}$ order statistic $X_{(j)}$ is. 
$$
\PP(X_{(j)} \leq x) = \sum_{k=j}^{n} {x \choose n} F(x) ^{k}(1 - F(x))^{n-k}
$$
The pdf of the $j^{th}$ order statistic $X_{(j)}$ is given by 
$$
f(X_{j}(x)) = \frac{n!}{(j-1)!(n-1)!}f(x)[F(x)]^{j-1}[1-F(x)]^{n-1}
$$
\end{theorem}
Let $X_{(1)},...,X_{(n)}$ denote the order statistics of a random sample, $X_1,...,X_n$, from continuous population with cdf $F_X(x)$ and pdf $f_X(x)$. Then the joint pdf of $X_{(i)}$ and $X_{(j)}$, $1 \leq i < j \leq n$, is 
\begin{align*}
    f_{X_{(i)},X_{(j)}}(u,v) = \frac{n!}{(i-1)!(j-1-i)! (n-j)!}
f_{X}(u)f_X(v)[F_X(u)]^{i-1} \\
\times [F_X(u) - F_X(u)]^{j-1-i}[1-F_X(v)]^{n-j}
\end{align*}
 

