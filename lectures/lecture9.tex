\section{October 6, 2021}
\subsection{Hierarchical Models}
Thus far we have seen random variables which have single distributions, possibly depending on parameters. However, we can think of the parameter of a distribution as being a random variable, which itself has a distribution. 

\begin{example}[Binomial-Poisson hierarchy]
Classic example of  hierarchical model is the following: An insect lays a large number of eggs, each surviving with probability $p$. On average, how many eggs will survive? 
The ``large number" of eggs laid is a random variable, often taken to be Poisson$(\lambda)$. Furthermore, if we assume that each eggs survival is independent, we have Bernoulli trials. Therefore if $X =$ number of survivors and $Y=$ number of eggs laid, we have
$$
X|Y \sim \text{binomial}(Y,p)
$$
$$
Y \sim \text{Poisson}(\lambda),
$$
a hierarchical model.
\end{example}
\term{Solution:}\\
The random variable of interest, $X =$ number of survivors, has the distribution given by
\begin{align*}
    \PP(X=x) &= \sum_{y=0}^{\infty} \PP(X=x,Y=y) \\
             &= \sum_{y=0}^{\infty} \PP(X=x|Y=y) \PP(Y=y) \\
             &= \sum_{y=x}^{\infty} \left[\binom{y}{x}p^x(1-p)^{y-x} \right] \left[ \frac{e^{-\lambda \lambda^y}}{y!} \right] \\
             & \vdots \quad \text{(after some algebraic simplifications)} \\ 
             &= \frac{(\lambda p)^x e^{- \lambda}}{x!} e^{(1-p) \lambda}
\end{align*}
Thus, $X \sim$Poission($\lambda p$). Thus, any marginal inference on $X$ is with respect to a Poisson($\lambda p$) distribution, with $Y$ playing no part at all. Introduction $Y$ in the hierarchy was mainly to aid our understanding the model. 
\\
Now we can easily compute the expected value
$$
\EE[X]= \lambda p
$$
so, on average, $\lambda p$ eggs will survive. 
\\
\\
Sometimes calculations can be greatly simplified using the following theorem. Recall that $\EE(X|y)$ is a function of $y$ and $\EE(X|Y)$ is a random variable whose distribution depends on the value of $Y$.

\begin{theorem}
    If $X$ and $Y$ are random variable, then
    $$\EE(X) = \EE \EE(X|Y)$$
    and,
    $$
    \VV ar(X) = \VV ar(\EE (X|Y)) + \EE (\VV ar(X|Y))
    $$
\end{theorem}
\begin{proof}
    By definition $$\underbrace{\EE(X|Y) = \sum_{x}x\PP(X=x|Y=y)}_\text{Average of $X$ when we fix $Y=y$}.$$ 
    But $Y$ is a random variable, so if average over all realizations of $Y$, we have,
    $$\EE_{Y}\left(\EE_X(X|Y)\right) = \sum_{y}\underbrace{ \sum_{x}x\PP(X=x|Y=y)}_{\EE(X|Y)}\cdot\PP(Y=y)$$
    by the definition of joint density, we can re-write the equation,
     $$
     \implies \sum_{y}\sum_{x}x\PP(x,y) = \sum_{x}\sum_{y}x\PP(x,y) = \sum_{x}x\underbrace{\sum_{y}\PP(x,y)}_{\PP(X=x)} = \sum_{x}x\PP(X=x) = \EE(X).
     $$
\end{proof}